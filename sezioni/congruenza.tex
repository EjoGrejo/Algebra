\section{Classi di Congruenza}
Siano $a,b\in\mathbb{Z}$, $n\in\mathbb{N}$, $n>0$\\
Si dice che {\color{red} $a$ è \textbf{congruo} (o congruente) a $b$ modulo $n$ se}
$$n|(a-b)$$
Si scrive $a\equiv b\mod n$; oppure $a\equiv b (\mod n)$ oppure $a\equiv _n b$
\paragraph{NB} $a\equiv b\mod n \Longleftrightarrow$
\begin{tabular}{c}
    il resto della divisione\\
    di $a$ per $n$
\end{tabular}$=$
\begin{tabular}{c}
    il resto della divisione\\
    di $b$ per $n$
\end{tabular}
\paragraph{Dimostrazione} ipotesi: $a\equiv b\mod n$ tesi: i due resti sono uguali \\
divido $a$ per $n$: {\color{green} $a=nq_1+r_1$}, {\color{blue} $0\leq r_1<n$}\\
divido $b$ per $n$: {\color{green} $b=nq_2+r_2$}, {\color{blue} $0\leq r_2<n$}\\
So che $a\equiv b\mod n \Longrightarrow n|(a-b)$\\
Da $a-b\underset{{\color{blue}\uparrow}}{=}nq_1 +r_1-(nq_2+r_2)=n(q_1-q_2)+(r_1-r_2)$\\
{\color{blue} $a=nq_1+r_1\\ b=nq_2 +r_2$}\\\\
Si ottiene: $r_1-r_2=(a-b)-n(q_1-q_2)$
$$
\begin{cases}
    n|n(q_1-q_2)\\
    {\color{red} n|a-b}
\end{cases}
\Longrightarrow n| (a-b)-n(q_1-q_2)\Longrightarrow n|r_1-r_2
$$
{\color{red} Perché per ipotesi $a\equiv b\mod n$    }
$$
\textrm{se }r_1\geq r_2\Longrightarrow 
\begin{cases}
0\leq r_1-r_2<n\\
n|r_1-r_2
\end{cases}
\Longrightarrow 
r_1-r_2=0\Longrightarrow r_1=r_2
$$
        
    
$$
\textrm{se }r_2\geq r_1\Longrightarrow 
\begin{cases}
0\leq r_2-r_1<n\\
    n|(r_1-r_2)\Rightarrow n|(r_2-r_1)
\end{cases}
\Longrightarrow 
r_2-r_1=0\Longrightarrow r_2=r_1
$$
\paragraph{Viceversa}
\paragraph{Ipotesi} Considero\\
\begin{tabular}{c c}
    $a=nq_1 +r_1$ & $0\leq r_1<n$\\
    $b=nq_2 +r_2$ & $0\leq r_2<n$\\
    $r_2=r_1$
\end{tabular}
\paragraph{Tesi} $a\equiv b\mod n$
\paragraph{Dimostrazione} Voglio arrivare a dire che $n|(a-b)$\\
$
\begin{cases}
    a=nq_1+r_1\\
    r_1=r_2
\end{cases}
\Longrightarrow a = nq_1+r_2 \Longrightarrow \\
a-b = (nq_1+r_2)-(nq_2+r_2)= nq_1+\cancel{r_2}-nq_2-\cancel{r_2}=nq_1-nq_2=\\
=n(q_1-q_2)\Longrightarrow n|(a-b)
$
\paragraph{NB 2} {\color{red} Fisso $n \in \mathbb{N}$}\\
La relazione di congruenza gode delle seguenti proprietà:
\begin{enumerate}
    \item \textbf{è riflessiva}: $a\equiv a\mod n\forall a$\\
        (infatti $n|(a-a)=0$)
    \item \textbf{è simmetrica}: $a\equiv b\mod n\Longrightarrow b\equiv a\mod n$\\
        (infatti $n|(a-b)\Longrightarrow n|(b-a)$)
    \item \textbf{è transitiva}: 
        $
        \begin{cases}
            a\equiv b\mod n\\
            b\equiv c\mod n
        \end{cases}
        \Longrightarrow a\equiv c\mod n
        $\\
        Infatti 
        $
        \begin{cases}
            a\equiv b\mod n\Longrightarrow n|(a-b) \\
            b\equiv c\mod n\Longrightarrow n|(b-c)
        \end{cases}
        \Longrightarrow n|[(a-b)+(b-c)]=(a-c)\Longrightarrow\\
        \Longrightarrow a\equiv c \mod n 
        $\\
        Ogni relazione che dove delle proprietà 1., 2., 3. si dice una \\
        {\color{red}\textbf{relazione di equivalenza}.}\\
        Fissato $n\in\mathbb{N}$, $n>0$, $a_1, a_2, b_1, b_2\in\mathbb{Z}$
    \item 
        $
        \begin{cases}
            a_1\equiv b_1\mod n\\
            a_2\equiv b_2\mod n
        \end{cases}
        \Longrightarrow (a_1 + a_2)\equiv (b_1+b_2)\mod n\\
        $
        {\color{red} le congruenze modulo $n$ si possono ``sommare"}
    \item 
        $
        \begin{cases}
            a_1\equiv b_1\mod n\\
            a_2\equiv b_2\mod n
        \end{cases}
        \Longrightarrow a_1 \cdot a_2\equiv b_1\cdot b_2\mod n\\
        $
        {\color{red} le congruenze modulo $n$ si possono ``moltiplicare"}
        {\color{purple} PAOLO qui però ho copiato parecchio dalle slide vecchie}



\end{enumerate}
In generale 
\begin{enumerate}
    \item $\forall n\in\mathbb{N}$, $\forall a, k \in\mathbb{Z}$
        $$[a]_n=[a+kn]_n$$
    \item $c\in [a]_n\Longrightarrow [a]_n=[c]_n$
    \item In particolare (dividevo) $a$ per:\\
        $a=qn+r$ con $0\leq r< n$\\
        Si ha $[a]_n=[r]_n$\\
        Perché, essendo $r=a+n\cdot(-q)$, si ha che $r\in[a]_n$, quindi si può 
        usare $[z]$??? con $c=r$
\end{enumerate}
\paragraph{Def.} $a\in\mathbb{Z}, n\in\mathbb{Z}, n>0$, si chiama
\\{\color{red} 
\textbf{classe di congruenza} $a$ modulo $n$} e si indica $[a]_n$ oppure $[a]\mod n$\\
\begin{align*}
    [a]_n & = \textrm{ insieme di tutti i numeri interi che sono congrui ad }a \textrm{ modulo }n\\
    & =\{b\in\mathbb{Z}|b\equiv a\mod n\}\\
\end{align*}
\paragraph{NB 1} $\forall b\in\mathbb{N}$, $n>0$, $a,b\in\mathbb{Z}$\hspace{1cm}$[a]_n,[b]_n$\\
Voglio vedere che $[a]_n=[b]_n$ oppure che $[a]_n \cap [b]_n=\varnothing$ \footnote{
    $[a]_n$ e $[b]_n$, pensati come insiemi di numeri interi, sono \textbf{insiemi disgiunti}}
\color{blue}\\
Infatti o $[a]_n=[b]_n$

Oppure $[a]_n\neq[b]_n$. Suppongo $[a]_n\cap[b]_n\neq\varnothing$\\
$\Longrightarrow \exists c \in[a]_n\cap[b]_n \Longrightarrow
\begin{cases}
    c\in[a]_n\Longrightarrow [a]_n=[c]_n\\
    c\in[b]_n\Longrightarrow [b]_n=[c]_n\\
\end{cases}\\
\Longrightarrow \color{red}
[a]_n=[c]_n=[b]_n\Longrightarrow[a]_n=b_n$
è una \textbf {contraddizione}
\color{black}
\paragraph{NB 2} $\forall n$, $n>0$\\
Considero le classi di congruenza $[a]_n$ con $0\leq a<n$\\
%\textbf{ogni numero intero appartiene ad una di queste classi.} Infatti,\\
se $b\in\mathbb{Z}$, dividendo $b$ su $n$ si ha: $b=nq+r$ con $0\leq r<n
\Longrightarrow [b]_n=[r]_n\Longrightarrow b\in[r]_n$\\ Quindi 
$$\mathbb{Z}=[0]_n\cup[1]_n\cup[2]_n\cup...\cup [n-1]_n$$
$$\mathbb{Z}=\underset{0\leq a <n}{\bigcup} [a]_n$$
Queste classi sono a due a due \textbf{disgiunte}, l'insieme delle classi 
$[0]_n,[1]_n,...,[n-1]_n$ sono una \textbf{partizione} di $\mathbb{Z}$
\paragraph{Def.} L'insieme degli {\color{red} interi modulo $n$}, indicato con il simbolo
$\mathbb{Z}_n$ è:\\
$$\mathbb{Z}_n=\Big\{ [0]_n,[1]_n,[2]_n,...,[n-1]_n \Big\}$$
In $\mathbb{Z}_n$ si definiscono $+$ e $\cdot$ nel seguente modo:\\
%\begin{tabular}{c c}
%   $\forall[a]_n,[b]_n\in\mathbb{Z}
%\end{tabular}

DA QUI RIPRENDO LEZIONE LIVE 6    

\paragraph{Teorema 1} (*) ha soluzione $\Longleftrightarrow d=MCD(a,n)|b$\\
se $d|b$ una soluzione $x_0=\alpha q$ dove $
\begin{cases}
    d=\alpha a+bn\\
    b=\alpha q\textrm{ per cui }q=\frac{b}{d}
\end{cases}
$
\paragraph{Teorema 2} se (*) ha soluzione e $x_0$ è una soluzione allora 
l'insieme di \textbf{tutte} le soluzioni è:\\
$\{x_k=x_0+k\cdot \frac{n}{d}|k\in\mathbb{Z}\}$ si ripartiscomno nelle classi: $[x_0]_n, [x_1]_n,...,[x_{d-1}]_n$\\
{\color{red} ESERCIZI}
\begin{enumerate}
    \item $\underset{a}{2}x\equiv\underset{b}{5}\mod\underset{n}{8}$
        \begin{enumerate}
            \item Calcolo $d=MCD(a,n)=MCD(2,8)=2$
            \item $d|b$ {\color{purple}PAOLO}
        \end{enumerate}
    \item $\underset{x}{3}\equiv\underset{b}{4}\mod\underset{n}{7}$
        \begin{enumerate}
            \item Calcolo $d=MCD(a,n)=MCD(3,7)=1$
            \item $d|b$\hspace{1cm}$1|4$\\
                La congruenza ha $\infty$ numeri come soluzioni:\\
                $\{x_0+7k| x\in\mathbb{Z}\}=[x_0]_7$ dove $x_0$ è una particolare soluzione.\\
                Soluzione: \\
                $d=\alpha a +\beta n$\\
                $1=\alpha\cdot 3+ \beta \cdot 7$\\
                Bezout:\\
                $\underset{n}{7}=\underset{a}{3}\cdot\underset{q_1}{2}+\underset{r_1}{1}
                \Longrightarrow d=1$\\
                $\underset{d}{1}=\underset{\beta =1}{7}+3\cdot (-2)\Longrightarrow \alpha=-2$\\
                $4=7\cdot 4+3\cdot(-2)\cdot 4$\\
                Le soluzioni sono tutte nella classe\\ $[(-2)\cdot 4]_7 \Longrightarrow [-8]_7
                =[-8+7]_7=[-1]_7=[-1+7]_7=[6]_7$\\
                {\color{purple} PAOLO}
        \end{enumerate}
    \item $\underset{a}{2}x\equiv\underset{b}{10}\mod\underset{n}{12}$\\
        {\color{purple} PAOLO}
        La congruenza ha infiniti numeri interi come soluzioni, che si ripartiscono in $d=2$ classi
        di congruenza modulo $n=12$\\
        \begin{enumerate}
            \item calcolo $x_0$ (poi prendero anche $x_1=x_0+6$)\\
                $\underset{a}{2}x\equiv\underset{b}{10}\mod\underset{n}{12}$
                $d=\alpha\cdot2+\beta\cdot 12$\\
                $2=\alpha\cdot2+\beta\cdot 12$\\
                $\underset{a}{2}=\underset{n}{12}\cdot\underset{q_1}{0}+\underset{r_1}{2}
                \Longrightarrow 
                Continua
                $\\
                $\underset{d}{2}=\underset{a}{2}\cdot {\alpha}{1}+\underset{n}{12}\cdot \underset{\beta}{0}$ Moltiplico per $5=\frac{b}{d}$\\
                $\underset{b=10}{5\cdot2}=5\cdot 2\cdot 1+5\cdot \underset{n}{12}\cdot 0$
                Continua\\
                L'insieme delle soluzioni della congruenza è:\\
                $\big\{ 5+12k|k\in\mathbb{Z}\big\}\cup\big\{11+12k|k\in\mathbb{Z}\big\}$

        \end{enumerate}


\end{enumerate}

\subsection{Invertibili in $\mathbb{Z}_n$ e il loro calcolo}
$n\in\mathbb{Z}$, $n>0$, $a\in\mathbb{Z}$ si dice \textbf{invertibile modulo $n$}
se la congruenza $ax\equiv 1\mod n$ ha soluzioni.\\
{\color{blue} quindi $\Longleftrightarrow MCD(a,n)=d|b=1\Longleftrightarrow MCD(a,n)=1$\\
Si dice {\color{purple} PAOLO}.}
\paragraph{Def.} $n\in\mathbb{N}$, $n>0$\\
$[a]_n\in\mathbb{Z}_n$ si dice \textbf{invertibile} in $\mathbb{Z}_n$ se \\
$\exists [b]_n\in\mathbb{Z}_n$ tale che $[a]_n[b]-n=[1]_n$\\
In questo case $[b]_n$ si dice \textbf{un inverso} di $[a]_n$\\
$[a]_n=[1]_n$
$ax\equiv 1\mod n$
$d=MCD(a,n)=1$
Essendo $[b]_n$ \textbf{unico}
(Perché $d=1$) Allora $[b]_n$ è l'\textbf{inverso} di $[a]_n$
    {\color{purple} PAOLO}

\paragraph{Esempio 1} $6$ non è iunvertibile modulo $9$ perché $MCD(6,9)\neq 1$\\
($6x\equiv 1\mod 9$ non ha soluzioni)
\paragraph{Esempio 2} 4 è invertibile modulo $9$ perché $MCD(4,9)=1$\\
($4$ e $9$ sono coprimi)\\
$underset{a}{4} x\equiv \underset{b}{1}\mod \underset{n}{1}$ ha soluzione\\
$\exists [4]_q^{-1}$\\
Calcolo l'inverso di $[4]_q$, cioè calcolo $[4]_q^{-1}$\\
$\underset{1}{d}=\alpha\underset{4}{a}+\underset{1}{\beta}\underset{9}{n}$\\
$\underset{n}{9}=\underset{a}{4}\cdot\underset{q_1}{2}+\underset{r_1}{1}$\\
$1=9+4\cdot (-2)$
{color{red}$\mathbb{Z}_p$ (con $p$ un numero primo)}
Sia $p$ un numero primo e $[a]_p\in\mathbb{Z}_p$\\
Posso supporre $0\leq a < p$A\\
\begin{description}
    \item [se $a = 0$] allora $[a]_p=[0]_p$\\
        $\not\exists [b]_p|[0]_p[b]_p=[1]_p$\\
        $\exists [0]_p^{-1}$
    \item[se $a\neq 0$] {\color{red} Siccome $p$ è un numero primo}
        {\color{purple}PAOLO}
\end{description}
Di $\mathbb{Z}_p$ \textbf{tutti di elementi} $\neq [0]_p$ sono \textbf{invertibili}.\\
Quanti sono? Sono $p-1$\\
Il numero degli elemtni invertibili in $\mathbb{Z}_p$ è $p-1$\\
Quanti sono gli invertibili in $\mathbb{Z}_n$?\\
{\color{purple}PAOLO}
\subsection{La funzione di Eulero}
La funzione di Eulero $\phi$ li ``conta"\\
$\phi : \mathbb{N}\longrightarrow\mathbb{N}$\\
è definita da $\phi(n)=$il numero dei naturali $k$ tali che 
$
\begin{cases}
    0\leq k<n\\
    MCD(k,n)=1
\end{cases}
$
Se $p$ è un numero primo ({\color{purple}PAOLO}) $\phi(p)=p-1$
$$n=p_1^{\alpha_1}p_2^{a_2}...p_m^{a_m}\Longrightarrow\phi(n)=n(1-\frac{1}{p_1})(1-\frac{1}{p_2}...
(1-\frac{1}{p_m})$$
Finisci slide 
\subsection{Sistema di congruenze}
UN sistema di congruenze è \\
$a_1x\equiv c_1\mod m_1$\\
$a_2x\equiv c_2\mod m_2$\\
...\\
$a_kx\equiv c_k\mod m_k$\\
Dove $a_i,c_i\in\mathbb{Z}$ $i=1,...,k$\\
{\color{purple}PAOLO\\}
{\color{red} ``Risolvere"} il sistema significa
\begin{itemize}
    \item Dire se ha soluzioni oppure no
    \item nel caso le abbia, trovarle tutte
\end{itemize}
Un $x_0\in\mathbb{Z}$ è UNA SOLUZIONE del sistema se è \textbf{contemporaneamente soluzione} 
di \textbf{ogni congruenza} del sistema. 
    
\paragraph{NB 1} Se una congruenza non ha soluzioni allora l'intero sistema non ne ha.\footnote{
    come avviene in tutti i sistemi} \\
\paragraph{NB 2} Anche se tutte le congruenze del sistema hanno soluzione, non è detto che il sistema abbia soluzione.\\
Ad esempio\\

$
\begin{cases}
    x\equiv 1\mod 2\\
    x\equiv 0\mod 6
\end{cases}$
non ha soluzioni anche se ogni sua configurazione ha soluzioni

\newpage
\subsection{Il teorema cinese dei resti}
Il teorema cinese dei resti da una condizione \textbf{sufficiente} affinché \textbf{particolari} sistemi di congruenze abbiano soluzioni. \\\\
Dati $n_1, n_2,...,n_k\in\mathbb{N}, n_i>0$\hspace{1cm}$i=1,...,k$\\
{\color{red} a due a due coprimi}\footnote{cioè se $i\neq j$ allora 
$MCD(n_i, n_j)=1$}\\\\
$\forall b_1,b_2,...,b_k\in\mathbb{Z}\textrm{ si ha che }
\exists \textrm{ infinite soluzioni del sistema}$\\

$
\begin{cases}
    x\equiv b_1\mod n_1\\
    x\equiv b_2\mod n_2\\
    ...\\
    x\equiv b_k\mod n_k\\
\end{cases}
$
\begin{tabular}{c}
Esse si trovano tutte nella stessa classe\\ di congruenze modulo
$n=n_1\cdot n_2\cdot...\cdot n_k$
\end{tabular}
\paragraph{NB} La condizione che gli $n_1$ siano a due a due coprimi
non è una condizione neccessaria affinché il sistema abbia soluzioni:
\paragraph{Esempio 1}
$
\begin{cases}
    5x\equiv 3\mod \underset{n_1}{7}\\
    3x\equiv 6\mod \underset{n_2}{7}
\end{cases}
$
\begin{tabular}{c}
 $n_1=n_2\Longrightarrow MCD(n_1,n_2)\neq 0$ \\Però il sistema ha soluzione $[2]_7$
\end{tabular}
\paragraph{Esempio 2}
$
\begin{cases}
    x\equiv 0\mod \underset{n_1}{2}\\
    x\equiv 2\mod \underset{n_2}{4}
\end{cases}
$
\begin{tabular}{c}
    $MCD(n_1,n_2)\neq 0$\\
Però il sistema ha soluzione in $[2]_4$
\end{tabular}
\paragraph{Cominciamo a studiare Il caso $k=2$} 
$$
\begin{cases}
    A\rightarrow\\
    B\rightarrow
\end{cases}
\begin{cases}
    x\equiv b_1\mod n_1\\
    x\equiv b_2\mod n_ 2
\end{cases}
MCD(n_1,n_2)=1
$$
%$n=n_1\cdot n_2$\\\\
%Per il teorema cinese dei resti, $\exists x_2$ soluzione di $(*)$, 
%e le soluzioni di $(*)$ sono esattamente gli interi nell'insieme
%$$\big\{x_2+nk|k\in\mathbb{Z}\big\}=[x_2]_n$$
    \subsubsection{Metodo di Newton}
    \begin{enumerate}
        \item $x_1=b_1$
        \item Cerco $t_2\in\mathbb{Z}$ tale che $x_1+t_2n_1\equiv x_2$ sia soluzione di $B$\\
            Così cerco $t_2\in\mathbb{Z}$ tale che \\
            $b_1=t_2n_1\equiv b_2\mod n_2$\\
            $t_2n_1\equiv (b_2-b_1)\mod n_2$\\
            \footnotesize{dove $t_2$ è il numero intero che cerco in modo tale che:\\
            $x_2\equiv b_2\mod 4$ (siccome cerco $t_2$)\\
            $x_2=x_1+t_2n_1\equiv x-1=b_1\mod n_1$}
        \item $x_2$ è una soluzione di $\begin{cases}A\\B\end{cases}$
        \item Per il teorema cinese dei resti, le soluzioni del sistema sono esattamente tutti i numeri interi
            nella classe $[x_2]_n = \{ \}$ {\color{purple} PAOLO} 
    \end{enumerate}
\paragraph{Esempio}
$
\begin{cases}
    x\equiv \underset{b_1}{4}\mod \underset{n_1}{6}\\
    x\equiv \underset{b_2}{3}\mod \underset{n_2}{5}
\end{cases}
$\\
$MCD(n_1,n_2)=MCD(6,5)=1$ \\
Posso applicare il teorema dinese dei resti e concludere che il sistema ha infinite soluzioni: tutti i numeri in $[x_2]_30=\{x_2+30k|k\in\mathbb{Z}\}$
\begin{enumerate}
    \item $x_1=4$
    \item cerco $t_2\in\mathbb{Z}$ tale che 
        \begin{tabular}{c c}
            & \\
            $x_2=$ & $x_1+t_2n_1\equiv b_2\mod n_2$, ovvero\\
            & $4+t_2\cdot 6\equiv 3\mod 5$
        \end{tabular}\\
        Facendo i conti in $\mathbb{Z}_5$: $[4]_5+t_2[6]_5=[3]_5$\\
        $t_2\cdot 6\equiv 3-4 \mod 5$\\
        $6t_2\equiv -1\mod 5\Longrightarrow t_2\equiv 4\mod 5$
    \item ad esempio prendo $t_2=4\Longrightarrow \\
        \Longrightarrow x_2=x_1+t_2n_1=4+4\cdot 6=28$
\end{enumerate}
Per il teorema cinese dei resti tutte le soluzioni di $\begin{cases}A\\B\end{cases}$ 
sono gli interi nell'insieme 
$[28]_{30} = \{28+30k|k\in\mathbb{Z}\}$
\paragraph{Il caso $k=3$} Consideriamo \\

\begin{tabular}{c}
    $A\longrightarrow$\\
    $B\longrightarrow$\\
    $C\longrightarrow$
\end{tabular}
$
\begin{cases}
    x\equiv b_1\mod n_1\\
    x\equiv b_2\mod n_2\\
    x\equiv b_3\mod n_3
\end{cases}
$\\\\
E lo risolviamo col teorema cinese dei resti con l'ipotesi:\\\\
$MCD(n_1,n_2)=1$\\
$MCD(n_1,n_3)=1$\\
$MCD(n_2,n_3)=1$\\\\
Per trovare $x_3$:
\begin{enumerate}
    \item Scelgo una soluzione di $A: x_1=b_1$
    \item Cerco $t_2\in\mathbb{Z}$ tale che $x_2=x_1+t_2n_1
        \equiv b_2\mod n_2$
    \item $x_2$ è soluzione di $\begin{cases}A\\B\end{cases}$
    \item Cerco $t_3\in\mathbb{Z}$ tale che $x_2+t_3(n_1\cdot n_2)=x_3$
        $x_3$ è soluzione di 
        $
        \begin{cases}
            A\\B\\C
        \end{cases}
        $\\\\
        $\underset{\mod n_1}{x_3}\equiv x_2$ è soluzione di $A$\\
        $\underset{\mod n_2}{x_3}\equiv x_2$ è soluzione di $B$
        $$n=n_1\cdot n_2\cdot n_3$$
    \item $x_3$ è una soluzione del sistema $\begin{cases}A\\B\\B\end{cases}$
\end{enumerate}
Per il teorema cinese dei resti la soluzione del $(*)$ sono i numeri interi nell'insieme $\{x_2+nk|k\in\mathbb{Z}\}$

\paragraph{Esempio 2} considero\\

$
\begin{cases}
    x\equiv 10\mod \underset{n_1}{11} \\
    x\equiv 5\mod \underset{n_2}{6} \\
    x\equiv 10\mod \underset{n_3}{7} \\
\end{cases}
$
\begin{tabular}{c}
    $MCD(11,6)=1$\\
    $MCD(11,7)=1$\\
    $MCD(6,7)=1$
\end{tabular}
$$n=11\cdot 6\cdot 7= 462$$
\begin{enumerate}
    \item $x_1=10$
    \item Cerco $t_2\in\mathbb{Z}$ tale che $x_2=x_1+t_2n_1\equiv b_2\mod n_2$\\
        $10=t_2\cdot 11\equiv 5\mod n$\\
        $11t_2\equiv 5-10 \mod 6$\\
        ${\color{blue}11}t_2\equiv {\color{red}-5} \mod 6$\\
        {\tiny
        {\color{blue}$[1]_6=[5]_6$}\\
        {\color{red}$[-5]_6=[1]_6$}
        }
        {\color{purple}PAOLO, e anche bello grosso}
    \item Cerco $t_3\in\mathbb{Z}$ tale che  $x_3=x_2+t_3(n_1\cdot n_2)$ 
        sia soluzione di $C$: $x\equiv 5\mod 7$\\
        $x_2+t_3(n_1\cdot n_2)\equiv 5\mod 7$\\
        $65+t_3(11\cdot 6)\equiv 5\mod 7 $\\
        $66t_3\equiv -60\mod 7 $\
        $3t_3\equiv 3\mod 7$\\\\
        \begin{align*}
            x_3 & =x_2+t_3\cdot n_1\cdot n_2\\
                & =65+1\cdot 11\cdot 6\\
                & =65+66=131
        \end{align*}
        {\color{purple} PAOLO}
\end{enumerate}
In generale se $k\geq 4$ e
$
\begin{cases}
    x\equiv b_1\mod n_1\\
    x\equiv b_2\mod n_2\\
    x\equiv b_k\mod n_k\\
\end{cases}
$ Con $MCD(n_i,n_j)=1$ $\forall i\neq j $\\
Itero di procedimento 
\begin{itemize}
    \item $x_1=b_1 $ è una soluzione di $1$
    \item impongo che $x_1+n_1t_2=x_2$ Sia soluzione di $2$
        {\color{purple}PAOLO} Cerco $t_2$...
    \item Impongo che  $x_2+n_1n_2t_3=x_3$ sia soluzione di $3$\\
        (Cerco $t_3\in\mathbb{Z}$ tale che ...) allora $x_3$ è soluzione di 
        $\begin{cases}
            1\\2\\3
        \end{cases}$
    \item Impongo che  $x_3+n_1n_2n_3t_4=x_3$ sia soluzione di $4$\\
        (Cerco $t_4\in\mathbb{Z}$ tale che ...) allora $x_4$ è soluzione di 
        $\begin{cases}
            1\\2\\3\\4
        \end{cases}$
\end{itemize}
{\color{purple} PAOLO}\\\\

Torniamo al caso $k=2$
$    \begin{cases}
    x\equiv b_1\mod n_1\\
    x\equiv b_2\mod n_2\\
\end{cases}$
Metodo di Lagrange $MCD(n_1,n_2)=1$

Da $MCD(n_1,n_2)=1$, usando Bezout trovo:
$\alpha_1, \alpha_2\in\mathbb{Z}$ tali che 
$$\alpha_1n_1+\alpha_2n_2=1$$
Allora $z=\alpha_1n_1b_2+\alpha_2n_2b_1$ è una {\color{purple}PAOLO}
..\\
..\\
..\\    
$z=\alpha_1n_1b_2+\alpha_2n_2b_1$\\
$z\equiv b_1\mod n_1$
$a_1n_1+\alpha_2n_2\Longrightarrow \alpha_2n_2=1-\alpha_1n_1$
\begin{align} 
    z&=\alpha_1n_1b_2+(1-\alpha_1n_1)b_1\\
     &=\alpha_1n_1b_2
\end{align} 
{\color{purple} \\\\PAOLO, c'è da finire la slide\\\\}


$
\begin{cases}
    x\equiv\underset{b_1}{4}\mod \underset{n_1}{6}\\
    x\equiv\underset{b_2}{4}\mod \underset{n_2}{6}\\
\end{cases}
$

$MCD(\underset{n_1}{6},\underset{n_2}{5})=1$ cerco $\alpha_1,\alpha_2\in\mathbb{Z}|$
{\color{purple} \\\\PAOLO\\\\}

\subsection{Ridurre un generico sistema di congruenze}
Vediamo come ``ridurre", se si può, un generico sistema di congruenze:

    $\underset{a_i, c_i \in\mathbb{Z}, m_i\in\mathbb{N}, m_i>0}
    {\begin{cases}
        a_1x\equiv c_1\mod m_1\\
        a_2x\equiv c_2\mod m_2\\
        ...\\
        a_kx\equiv c_k\mod m_k
    \end{cases}}$
ad un sistema nella forma 
$
\underset{b_i\in\mathbb{Z}, n_i\in\mathbb{Z}, n_i>0}
{\begin{cases}
    x\equiv b_1\mod n_1\\
    x\equiv b_2\mod n_2\\
    ...\\
    x\equiv b_k\mod n_k
\end{cases}}
$\\\\
\textbf{Ridurre} significa ``sostituire con un sistema equivalente"\\
\textbf{Equivalente} significa ``con le stesse soluzioni" 

%Lezione 

\paragraph{Motivazione} Abbiamo\\

{\color{blue}
    \begin{tabular}{c}
        $A\rightarrow$\\
        $B\rightarrow$
    \end{tabular}
}
$
\begin{cases} 
    2x\equiv 4\mod 8 \\
    3x\equiv 6\mod 9 
\end{cases} 
$\\
$A=MCD(2,8)=d=2|4
\begin{cases}
    [2]_8\hspace{0.5cm}2\cdot 2=4\equiv 4\mod 8\\
    [6]_8\hspace{0.5cm}2\cdot 6=12\equiv 4\mod 8
\end{cases}
$\\
$
A
\begin{cases}
    x\equiv 2\mod 8 \hspace{0.5cm} {\color{green} C}\\
    x\equiv 6\mod 8 \hspace{0.5cm} {\color{green} D}\\
\end{cases}
$\\
$
B: MCD(3,9)=d=3|6
\begin{cases}
    [2]_9 \hspace{.5cm} 3\cdot 2=6\equiv 6\mod 9\\
    [5]_9 \hspace{.5cm} 3\cdot 5=15\equiv 6\mod 9\\
    [8]_9 \hspace{.5cm} 3\cdot 8=24\equiv 6\mod 9\\
\end{cases}
$\\
$
B
\begin{cases}
    x\equiv 2\mod 9 \hspace{0.5cm} {\color{green} E}\\
    x\equiv 5\mod 9 \hspace{0.5cm} {\color{green} F}\\
    x\equiv 8\mod 9 \hspace{0.5cm} {\color{green} G}
\end{cases}
$\\
\\
Quindi le soluzioni di $\begin{cases}A\\B\end{cases}$ sono l'unione delle soluzioni
di 6 sistemi:
$$
\begin{cases}
    C\\E
\end{cases}\cup
\begin{cases}
    C\\F
\end{cases}\cup
\begin{cases}
    C\\G
\end{cases}\cup
\begin{cases}
    D\\E
\end{cases}\cup
\begin{cases}
    D\\F
\end{cases}\cup
\begin{cases}
    D\\G
\end{cases}
$$
E noi vorremmo non dover risolvere sei sistemi.
\paragraph{Passaggio 1} Calcolo $d_i=MCD(a_i,m_i)$ $\forall i=1,...,k$
%
% non so bene come contestualizzare: 
%
% $i$-esima congruenza di $(*)\longrightarrow a_ix\equiv c_i\mod m_i
\begin{itemize}
    \item $\exists d_i$ tale che $d_i\not| c_i$ allora $a_ix\equiv c_i \mod m_i$ Non ha soluzioni, allora $(*)$ non ha soluzioni.
    \item se $d_i|c_i$ $\forall i=1,...,k$ allora ogni congruenza di $(*)$ ha soluzione e 
        \begin{itemize}
            \item se $d_i=1$ \textbf{mantengo} la congruenza $a_ix\equiv c_i \mod m_i$
            \item se $d_i\neq1$ \textbf{sostituisco} la congruenza $a_ix\equiv c_i \mod m_i$\\
                con la congruenza
                $$\frac{a_i}{d_i}x\equiv\frac{c_i}{d_i}\mod\frac{m_i}{d_i}$$
        \end{itemize}
\end{itemize}
\paragraph{NB 1} La congruenza $\frac{a_i}{d_i}x\equiv\frac{c_i}{d_i}\mod\frac{m_i}{d_i}$ è \textbf{equivalente} alla congruenza $a_ix\equiv c_i \mod m_i$
\paragraph{NB 2} La congruenza $a_ix\equiv c_i \mod m_i$\\

{\color{blue}
    Infatti\\\\
    Sia $z\in\mathbb{Z}$
    \begin{center}
        \begin{tabular}{|c|}
            \hline
            $z$ è soluzione\\
            di $a_ix\equiv c_i \mod m_i$\\
            \hline
        \end{tabular}
        $\Longleftrightarrow$
        \begin{tabular}{|c|}
            \hline
            $\exists k\in\mathbb{Z}$ tale che \\
            $a_iz=c_i+m_ik$\\                             %questo è m_ik?
            \hline
        \end{tabular}
    \end{center}
}
\begin{tabular}{c}
    {\color{green}divido per $d_i$}\\
    {\color{green}$\Longrightarrow$}\\
    {\color{orange}$\Longleftarrow$}\\
    {\color{orange}moltiplico per $d_i$}\\
\end{tabular}
{\color{blue}
    \begin{tabular}{|c|}
        \hline
            $\exists k\in\mathbb{Z}$ tale che\\
            $\frac{a_i}{d_i}z=\frac{c_i}{d_i}+\frac{m_i}{d_i}k$\\
        \hline
    \end{tabular}
    $\longleftrightarrow$
    \begin{tabular}{|c|}
        \hline
            $z$ è soluzione di \\
            $\frac{a_i}{d_i}x\equiv \frac{c_i}{d_i}\mod \frac{m_i}{d_i}$\\
        \hline
    \end{tabular}
}

\paragraph{NB 3} Siccome $d_i=MCD(a_i, m_i)$ allora
$$MCD(\frac{a_i}{d_i},\frac{m_i}{d_i})=1$$
Quindi le soluzioni della congruenza $\frac{a_i}{d_i}x\equiv \frac{c_i}{d_i}\mod\frac{m_i}{d_i}$
stanno tutte in un'unica classe di congruenza modulo $\frac{m_i}{d_i}$\\
Alla fine del \textbf{passaggio 1} ottengo che $(*)$ non ha soluzioni, oppure che $(*)$ è equivalente a 

$$(**)
\begin{cases}
    \frac{a_1}{d_1}x\equiv \frac{c_1}{d_1}\mod \frac{m_1}{d_1}\\
    :\\
    :\\
    \frac{a_k}{d_k}x\equiv \frac{c_k}{d_k}\mod \frac{m_k}{d_k}
\end{cases}
$$
\paragraph{Passaggio 2} Risolvo ciascuna congruenza di $(**)$ 
$$\frac{a_i}{d_i}x\equiv \frac{c_i}{d_i}\mod\frac{m_i}{d_i} \Longrightarrow x\equiv {\color{orange}b_i}\mod \frac{m_i}{d_i}$$
{\color{orange}
    Dove $[b_i]_{\frac{m_i}{d_i}}=\{ b_i+\frac{m_i}{d_i}t|t\in\mathbb{Z}\}$ è \textbf{l'insieme delle soluzioni della congruenza}
}\\\\
Posto $n_i=\frac{m_i}{d_i}$ ottengo un sistema
$$
(***)
\begin{cases}
    x\equiv b_1\mod n_1\\
    x\equiv b_2\mod n_2\\
    ...\\
    ...\\
    x\equiv b_k\mod n_k\\
\end{cases}
$$
{\color{red} SE} $MCD(n_i,n_j)=1$ $\forall i\neq j $ posso applicare il Teorema cinese dei resti. In tal caso:
\paragraph{Passaggio 3} Con newton trovo $x_k$ una particolare soluzione di $(***)$ e per il teorema cinese dei resti l'insieme di tutte le soluzioni $(***)$,
{\color{red} e quindi anche di $(*)$} è $[x_k]_n=\{x_k+nt|t\in\mathbb{Z}\}$\\
$$\textrm{dove }n=n_1\cdot n_2\cdot...\cdot n_k$$



\subsection{Esercizio tipo} Risolvere il sistema\\

$
\begin{cases}
    \underset{a_1}{3}x\equiv \underset{c_1}{4}\mod \underset{m_1}{5}\\
    \underset{a_2}{2}x\equiv \underset{c_2}{4}\mod \underset{m_2}{6}
\end{cases}
$

\paragraph{Passaggio 1} 
\begin{tabular}{c}
    $a_1=MCD(a_1,m_1)=MCD(3,5)=1|4=c_1$\\
    $a_2=MCD(a_2,m_2)=MCD(2,6)=2|4=c_2$
\end{tabular}
\\
\\
$a_1=1\Longrightarrow $ mantengo $3x\equiv 4\mod 5$\\
$a_2=2\neq1 $ \textbf{sostituisco}  $2x\equiv 4\mod 6$\\
Con $\frac{2}{\color{red} 2}x\equiv \frac{4}{\color{red} 2} \mod \frac{6}{\color{red} 2}{\color{red} : x\equiv 2\mod 3}$
$$
\textrm{arrivo a }(**)
\begin{cases}
    3x\equiv 4\mod 5\\
    x\equiv 2\mod 3
\end{cases}
$$

\paragraph{Passaggio 2} Risolvo ciascuna congruenza {\color{purple} PAOLO}\\  %difficoltà nel decifrare la slide 

$\underset{a}{3}x\equiv\underset{b}{4}\mod \underset{n}{5}$\\\\
$d=MCD(a,n)=1|4=b$\\
$d=1=\alpha a+\beta n $\\
$1={\color{red} \alpha}+\beta \cdot 5$\\
{\color{red} \scriptsize{$\alpha = 2\\ x_0=\alpha q=2\cdot 4 = 8 $ }}\\
{\color{blue}
    $
    \underset{n}{5}=\underset{a}{3}\cdot\underset{q_1}{1}+\underset{r_1}{2}\Longrightarrow {\color{orange} 2=5+3\cdot(-1)}\\
    \underset{a}{3}=\underset{r_1}{2}\cdot\underset{q_2}{1}+\underset{r_2}{1}\\
    $
} 
{\color{purple}
        $\Longrightarrow$
}
\begin{tabular}{c}
    {\color{purple} $1=3+3\cdot(-1) = 3+(-1)\big[5+3\cdot(-1)\big]=$}\\
    {\color{purple} $=3+(-1)\cdot 5 + 3=3 \cdot 2 + 5 \cdot (-1)$}
\end{tabular}
$3x\equiv 4\mod 5$\\
$[8]_5=[8-5]_5=[3]_5$\\
Sostituisco $3x\equiv 4\mod 5$ con $x\equiv 3\mod 5$\\
Per puro caso la congruenza $x=2\mod 3$ è già risolta. 

$$(***)
\begin{cases}
    x\equiv \underset{b_1}{3}\mod \underset{n_1}{5}\\
    x\equiv \underset{b_2}{2}\mod \underset{n_2}{3}
\end{cases}
$$
Siccome $MCD(n_1,n_2)=MCD(5,3)=1$, \\
Allora 
{\color{blue}
    posso applicare il teorema cinese dei resti e concludere che 
}
$(***)$ e quindi anche il sistema da cui sono partit\textschwa{ }ha infinite soluzioni (numeri interi)
tutte nella stessa classe di congruenza modulo 

$$n=n_2\cdot n_2 = 5\cdot 3 = 15$$

\paragraph{Passaggio 3} Trovo $x_2$ una particolare soluzione di $(***)$
\begin{description}
    \item[1° Modo] per trovare
        $x_2
        \begin{cases}
            x\equiv\underset{b_1}{3}\mod\underset{n-1}{5}\\
            x\equiv\underset{b_2}{2}\mod\underset{n-2}{3}
        \end{cases}
        $
        \begin{enumerate}
            \item $x_1=3$
            \item cerco $t_2\in\mathbb{Z}$ tale che $x_2=x_1+t_2n_1 \equiv 2\mod 3$\\
                {\color{blue}
                    $x_2\rightarrow 3+t_2\cdot 5$} $\equiv 2 \mod 3$
        \end{enumerate}
        $5t_2\equiv (2-3)\mod 3$\\
        $5t_2\equiv -1\mod 3\equiv 2 \mod 3$A\\
        $[5]_3 = [2]_3 \rightarrow 5t_2 = 2t_2$\\\\
        $2t_2\equiv 2\mod 3$\\
        Ad esempio $t_2=1$
        {\color{red}
            $x_2=3+1 \cdot 5 = 3 + 5 = 8 $\\
            \textbf{tutte le soluzioni del (*) sono 
            $\pmb{[8]_5=\{8+15k|k\in\mathbb{Z}\}}$} 
        }
    \item[2° Modo] per trovare $x_2=z$
        $MCD(n_1,n_2)=1$ $\exists \alpha_1,\alpha_2\in\mathbb{Z}$ tale che \\
        $\alpha_1n_1+\alpha_2n_2=1$\\
        {\color{red}
            $\underset{-1}{\alpha_1}$
        }
        $\cdot 5 +$
        {\color{red} 
            $\underset{2}{\alpha_2}$
        }
        $\cdot 3 = 1$ 
        \\
        \\
        $z=$
        {\color{red}
            $\underset{-5}{\alpha_1n_1}$
        }
        $b_2+$
        {\color{red}
            $\underset{6}{\alpha_2n_2}$
        }
        $b_1=\\
        =-5\cdot 2+6\cdot 3\\
        =-10+18=8$
        $$\pmb{[z]_n=[8]_{15}=\{8+15k|k\in\mathbb{Z}\}}$$
\end{description}













