\section{Spazi vettoriali reali e complessi}
Sia $K\in\{\mathbb{R},\mathbb{C}\}$

Uno spazio vettoriale su $K$
\begin{itemize}
    \item se $K=\mathbb{C}$ dirò uno spazio vettoriale (complesso)
    \item se $K=\mathbb{C}$ dirò uno spazio vettoriale reale
\end{itemize}
è un insieme \textbf{non vuoto} $V$ su cui sono definite due operazioni

\begin{description}
    \item[addizione:] $V \times V \longrightarrow v$
    \item[prodotto di elementi di V per uno scalare:] $k \times v = v$
\end{description}
che verificano le seguenti condizioni:
\begin{itemize}
    \item $\forall \vec{u}, \vec{v}, \vec{w} \in V$ (gli elementi di $V$ si chiamano \textbf{vettori})
    \item $\forall \alpha, \beta \in K$ (gli elementi di $K$ si chiamano \textbf{scalari})
\end{itemize}
\begin{enumerate} 
    \item $\vec{u} + (\vec{v} + \vec{w})=(\vec{u} + \vec{v}) + \vec{w}$ {\hspace{1cm}}+ associativa
    \item $\vec{u}+\vec{v} = \vec{v} + \vec{u}$ {\hspace{1cm}}+ commutativa
    \item $\alpha(\beta\vec{v})=(\alpha\beta)\vec{v}$
    \item $1\cdot \vec{v} = \vec{v}$
    \item $(\alpha + \beta)\vec{v}=\alpha\vec{v} + \beta\vec{v}$
    \item $\alpha (\vec{u} + \vec{v})= \alpha\vec{u} + \alpha\vec{v}$
    \item $\exists\vec{0}\in V$ tale che $\vec{v}+\vec{0}=\vec{v}$
    \item $\forall \vec{v} \in V \exists \vec{w}\in V$ tale che $\vec{v}+\vec{w}=\vec{0}$, 
        \hspace{1cm} $\vec{w}$ 
        si indica con $-\vec{v}$
\end{enumerate}
\paragraph{Esempi}
\begin{enumerate}
    \item $\mathbb{R}^2, \mathbb{R}^n, \mathbb{R}_n, M_{m\times n}(\mathbb{R})$ 
        sono spazi vettoriali reali \\
    $\mathbb{C}^2, \mathbb{C}^n, \mathbb{C}_n, M_{m\times n}(\mathbb{C})$ sono spazi vettoriali 
        (complessi)
    \item $a,b\in\mathbb{R}\hspace{1cm} a<b\hspace{1cm} [a,b]=\{x\in\mathbb{R}|a\leq x \leq b\}$\\
    $f:[a,b]\longrightarrow \mathbb{R}$, \hspace{1cm} $\{f|f:[a,b]=\mathbb{R}\}=\mathcal{F}([a,b])$ 
        \footnote{dove $f$ è l'insieme delle funzioni definite in $[a,b]$ 
        ed a valori in $\mathbb{R}$}
$\mathcal{F}([a,b])$ è uno spazio vettoriale reale rispetto a:
\begin{itemize}
    \item $\pmb{+}$ : $\mathcal{F}([a,b])\times \mathcal{F}([a,b])
        \longrightarrow \mathcal{F}([a,b])$  \\
        $(f,g)\longrightarrow f + g : [a,b]\longrightarrow \mathbb{R}$\\
        $(f+g)(x)\overset{def}{=}{\color{blue}f(x)+g(x)}\hspace{1cm}\forall x \in [a,b]$
    \item $\pmb{\cdot}$ : $\mathbb{R}\times\mathcal{F}([a,b])\longrightarrow \mathcal{F}([a,b])$\\
        $(\alpha, f)\longrightarrow \alpha f: [a,b]\longrightarrow \mathbb{R}$\\
        $(\alpha f)(x)\overset{def}{=}{\color{blue} \alpha\cdot f(x)}\hspace{1cm}\forall x\in [a,b]$
\end{itemize}
{\color{green}
    $\mathcal{C}([a,b]) = \{f \in \mathcal{F}([a,b])| f \textrm{ continue}\}$\\
    con le stesse operazioni di sopra è uno spazio vettoriale reale
}
    \item $\mathbb{R}[x] = $insieme dei polinomi a coefficienti reali \\
        è uno spazio vettoriale reale (rispetto a $+$ e $\cdot$ ) \\
    $\mathbb{C}[x] = $insieme dei polinomi a coefficienti complessi \\
        è uno spazio vettoriale (rispetto a $+$ e $\cdot$ ) \\
    \item $\mathbb{R}_n[x]=\{f(x)\in\mathbb{R}[x]|\deg f(x)\leq n\}$\\
        è uno spazio vettoriale reale
    \item $\{\vec{0}\}$ contiene un unico vettore, che chiamo $\vec{0}$\\\\
        $\vec{0} + \vec{0} = \vec{0}$\\
        $\alpha \in K \hspace{1cm} \alpha\cdot\vec{0} = \vec{0}$\\
        $k\in\{\mathbb{R}, \mathbb{C}\}$\\\\
        Rispetto a queste operazioni, $\{\vec{0}\}$ è un {\color{purple} PAOLO} vettore su $K$
\end{enumerate}
\paragraph{NB}
Sia $V$ uno spazio vettoriale su $K\in\{\mathbb{R},\mathbb{C}\}$\\ Allora
\begin{enumerate}
    \item $0 \cdot \vec{v} = \vec{0}$, $\forall \vec{v}\in V$
    \item $\alpha \cdot \vec{0} = \vec{0}$, $\forall \alpha\in K$
    \item {\color{red} Vale la legge di cancellazione per il prodotto per scalari}\\
        $\alpha\in k$, $\vec{v}\in V$ e $\alpha\vec{v}=\vec{0}
        \Longrightarrow \alpha = \vec{0}$ o $\vec{v}=\vec{0}$
    \item $-(\alpha\vec{v})=(-\alpha)\vec{v}=\alpha(-\vec{v})$, 
        $\forall \alpha\in K$, $\forall\vec{v}\in V$

\end{enumerate}


\subsection{Sottospazi di spazi vettoriali}
Sia $V$ uno spazio vettoriale in $K\in\{\mathbb{R}, \mathbb{C}\}$
\paragraph{Def} Un sottoinsieme $U$ di $V$ si dice un \textbf{sottospazio vettoriale}
(o semplicemente 
un \textbf{sottospazio}) di $V$ se sono soddisfatte le seguenti condizioni:
\begin{enumerate}
    \item $\vec{0}\in U$
    \item $\vec{u}_1+\vec{u}_2\in U$, $\forall \vec{u}_1,\vec{u}_2\in U$ 
        ($U$ è chiuso alla \textbf{somma})
    \item $\alpha\vec{u}\in U$, $\forall\vec{u}\in U$, $\forall\alpha\in K$ 
        ($U$ è chiuso al \textbf{prodotto per scalari})
\end{enumerate}
\paragraph{NB 1} Sia $V$ uno spazio vettoriale su $K \in \{\mathbb{R}, \mathbb{C}\}$ \\
Sia $U$ un sottoinsieme di $V$. Allora\\\\

$\begin{bmatrix}
    U \textrm{ soddisfa le condizioni}\\
    \pmb{\vec{0}\in U}\\
    \vec{u}_1=\vec{u}_2\in U, \forall \vec{u}_1, \vec{u}_2\in U\\
    \alpha \vec{u}\in U, \forall \vec{u}\in U, \forall \alpha \in K
\end{bmatrix}
\Longleftrightarrow
\begin{bmatrix}
    U \textrm{ soddisfa le condizioni}\\
    \pmb{U \neq \varnothing}\\
    \vec{u}_1=\vec{u}_2\in U, \forall \vec{u}_1, \vec{u}_2\in U\\
    \alpha \vec{u}\in U, \forall \vec{u}\in U, \forall \alpha \in K
\end{bmatrix}$\\\\\\
``$\Longrightarrow$" ovvia: $\vec{0}\in U\Longrightarrow U\neq \varnothing$\\
``$\Longleftarrow$" $U \neq \varnothing \Rightarrow \exists \vec{u}\in U 
\Rightarrow \alpha\vec{u}\in U \forall \alpha \in k \Rightarrow 0\cdot \vec{u}\in U$

\paragraph{NB 2} {\color{red}
    Sia $V$ uno spazio vettoriale su $K \in\{\mathbb{R}, \mathbb{C}\}$\\
    Se $U$ è un sottospazio di $V$, allora $U$ è uno spazio vettoriale if 
    (con le operazioni $+$ e $\cdot$ che si ottengono restringendo quelle di $V$)
}

\paragraph{Esempio 1} $V=\mathbb{R}^3$ è uno spazio vettoriale reale $(K=\mathbb{R})$\\
$$U=\Bigg\{
    \begin{bmatrix}
        a\\
        0\\
        b
    \end{bmatrix}
    | a,b\in\mathbb{R} 
    \Bigg\}
\subseteq V \hspace{1cm} U \textrm{ è sottospazio di }V?$$
\begin{enumerate} 
    \item $\vec{0}\in U$ \footnote{
        $\begin{bmatrix}
            0\\
            0\\
            0
        \end{bmatrix}
        =0\in V=\mathbb{R}^3$
        }
        \hspace{2cm}
        {\color{blue}
        $
            \underset{?}{\exists} a,b\in\mathbb{R}|
            \begin{bmatrix}
                0\\
                0\\
                0
            \end{bmatrix}
            =
            \begin{bmatrix}
                a\\
                0\\
                b
            \end{bmatrix}
        $
        \\
        Si: $a=0$, $b=0$
}\\\\
        Quindi $\vec{0}\in U$
    \item $\vec{u}_1+\vec{u}_2\in U$ $\forall \vec{u}_1,\vec{u}_2\in U$\\
        $ 
        \vec{u} = 
        \begin{bmatrix}
            a_1\\
            0\\
            b_1
        \end{bmatrix}
        $
        per opportuni $a_1,b_1\in\mathbb{R} $
        \\
        $ 
        \vec{u}_2= 
        \begin{bmatrix}
            a_2\\
            0\\
            b_2
        \end{bmatrix}
        $ 
        per opportuni $a_2,b_2\in\mathbb{R} $
        \\
        $ 
        \vec{u}_1+\vec{u}_2= 
        \begin{bmatrix}
            a_1\\
            0\\
            b_1
        \end{bmatrix}
        +
        \begin{bmatrix}
            a_2\\
            0\\
            b_2
        \end{bmatrix}
        =
        \begin{bmatrix}
            a_1 + a_2\\
            0\\
            b_1 + b_2
        \end{bmatrix}
        \underset{?}{\in}\mathbb{R}
        $\\

        $
        \underset{?}{\exists}a,b \in\mathbb{R}| 
        \begin{bmatrix}
            a_1+a_2\\
            0\\
            b_1+b_2
        \end{bmatrix}
        = 
        \begin{bmatrix}
            a\\0\\b
        \end{bmatrix}
        $
        Sì 
        $ 
        \begin{cases}
            a=a_1+a_2\\
            b=b_1+b_2\\
        \end{cases}
        $
        \\\\
        \textbf{Quindi $U$ è chiuso alla somma}
    \item $\alpha \vec{u} \underset{?}{\in} U$ $\forall\vec{u}\in U$ $\forall\alpha\in\mathbb{R}$
        $
        \vec{u}=
        \begin{bmatrix}
            a\\0\\b
        \end{bmatrix}
        \hspace{1cm}
        \alpha\vec{u}=\alpha
        \begin{bmatrix}
            a\\0\\b
        \end{bmatrix}
        = 
        \begin{bmatrix}
            \alpha a\\0\\\alpha b
        \end{bmatrix}
        \underset{?}{\in}U$         \\
        $ 
            \underset{?}{\exists}a^*, b^* \in\mathbb{R}\Big|
            \begin{bmatrix}
                \alpha a\\0\\ \alpha b
            \end{bmatrix}
            = 
            \begin{bmatrix}
                a^*\\0\\b^*
            \end{bmatrix}
        $\\
        Sì: $
        \begin{cases}
            a^* =\alpha a\\
            b^*=\alpha b
        \end{cases}
            $
        Quindi $U$ è chiuso al prodotto per scalari\\
        Da \textbf{1.}, \textbf{2.} e \textbf{3.} concludo che $U$ è 
        un sottospazio di $\mathbb{R}^3$
\end{enumerate} 
\paragraph{Esempio 2} Sia $A\in M_{m\times n}(\mathbb{C})$ un insieme lineare
del tipo 
$$A\vec{x}=\vec{0}$$
si chiama \textbf{un sistema lineare omogeneo}
(ossia tale che il vettore dei termini noti sia $\vec{0}$), dove: 
\begin{itemize}
    \item $A= m\times n$
    \item $\vec{x} = n\times 1$
    \item $\vec{0} = m\times 1$
\end{itemize}

Un sistema lineare omogeneo ha sempre soluzioni, ad esempio la soluzione
nulla
\\{\color{purple}PAOLO}\\

Sia A$\in M_{m\times n}(\mathbb{C})$

\begin{center}
    $N($A$)=$
    \begin{tabular}{c}
        insieme delle\\
        soluzioni del \\
        sistema lineare\\
        omogeneo\\
        A$_{m\times n}\vec{x}_{n\times 1}=\vec{0}_{m\times 1}$
    \end{tabular}
    $=\{\vec{v}\in\mathbb{C}^n|$A$\vec{v}=\vec{0}\}$
\end{center}

$N($A$)$ è un sottoinsieme di $\mathbb{C}^n$\\
Proviamo che $N($A$)$ è un sottospazio di $\mathbb{C}^n$\\
(Quindi $N($A$)$ è a sua volta uno spazio vettoriale) \\
Chiamiamo $N($A$)$ lo \textbf{spazio nullo} della matrice A 
\begin{enumerate}
    \item $ 
        \begin{bmatrix}
            0\\
            :\\
            :\\
            0
        \end{bmatrix}
        =\vec{0}\underset{?}{\exists} N($A$)$\\\\
        Si: A$\vec{0}_{n\times 1}=\vec{0}_{m\times 1}$\\\\
        Quindi $\vec{0}_{n\times 1}= 
        \begin{bmatrix}
            0\\
            :\\
            :\\
            0
        \end{bmatrix}
        \in N($A$)$
    \item $\vec{u}_1, \vec{u}_2\in N($A$) \underset{?}{\Longrightarrow}
        \vec{u}_1+\vec{u}_2\in N($A$)$
        Per provare che $\vec{u}_1+\vec{u}_2\in N($A$)$ devo provare che 
        $
        \begin{cases}
            \vec{u}_1+\vec{u}_2\in\mathbb{C}^n\\
            \textrm{A}\times (\vec{u}_1+\vec{u}_2)=0
        \end{cases}
        $\\
        So che $\vec{u}_1+\vec{u}_2\in N($A$)$, quindi 
        $ 
        \begin{cases}
            \vec{u}_1\in\mathbb{C}^n\\
            \textrm{A} \vec{u}_1=0
        \end{cases}
        $ 
        e che $\vec{u}_2\in N($A$)$, quindi 
        $ 
        \begin{cases}
            \vec{u}_2\in\mathbb{C}^n\\
            \textrm{A} \vec{u}_2=0
        \end{cases}
        $ \\\\
        \footnotesize{
            $\vec{u}_1, \vec{u}_2\in\mathbb{C}^n 
            \Longrightarrow
            \vec{u}_1+\vec{u}_2\in\mathbb{C}^n$ 
            A$(\vec{u}_1+\vec{u}_2)=$A$\vec{u}_1+$A$\vec{u}_2=
            \vec{0}+\vec{0}\vec{0}$; 
            quindi $\vec{u}_1+\vec{u}_2\in N($A$)$
        }
    \item $
            \begin{cases}
                \vec{u}\in N(\textrm{A})\\
                \alpha \in \mathbb{C}
            \end{cases}
            \Longrightarrow
            \alpha\vec{u}\in n($A$)
            $\\
            Per provare che $\alpha\vec{U}\in N($A$)$ devo provare che $ 
            \begin{cases}
                \alpha\vec{u}\in\mathbb{C}^nA\\
                \textrm{A} \cdot (\alpha\vec{u})=0
            \end{cases}
            $\\
            So che $\vec{u}\in N($A$)$ quindi $ 
            \begin{cases}
                \vec{u}\in\mathbb{C}^n\\
                \textrm{A}\vec{u}=\vec{0}
            \end{cases}
            $\\
            $\vec{u}\in\mathbb{C}^n, \alpha\in\mathbb{C}\Longrightarrow 
            \alpha\vec{u}\in\mathbb{C}^n
            $\\
            A$\cdot (\alpha\vec{u})=\alpha\cdot($A$\vec{u})=\alpha\cdot\vec{0}
            =\vec{0}$ 
            Quindi $\alpha\vec{u}\in N($A$)$\\
            \textbf{1.} + \textbf{2.} + \textbf{3.} $\Longrightarrow N($A$)$ 
            \textbf{è un sottospazio di }$\pmb{\mathbb{C}^n}$ 
\end{enumerate}

\subsection{Insieme dei multipli di un vettore}
Siano $V$ uno spazio vettoriale su $K\in\{\mathbb{R},\mathbb{C}\}$ e 
$\vec{v}\in V$

$$\{\alpha\vec{v}|\alpha\in K\}=\textrm{insieme dei multipli di }V$$
\begin{center}
    Si indica $<\vec{v}>$ oppure Span$(\vec{v})$
\end{center}
\begin{enumerate}
    \item $\pmb{<\vec{v}>}$ \textbf{è un sottospazio di V}
        \begin{enumerate}
            \item $\vec{0}\in V$: $\vec{0}=0\cdot \vec{v}$ (prendo $a=0$)
            \item $\alpha_1\vec{v}+\alpha_2\vec{v}=(\alpha_1+\alpha-2)\vec{v}$ 
                La somma di due multipli di $\vec{V}$ è un multiplo di $\vec{v}$
            \item $\beta(\alpha\vec{v})=(\beta\alpha)\vec{v}$
                Il prodotto di $\beta$ per un multiplo di $\vec{v}$ è un multiplo di $\vec{v}$
        \end{enumerate}
    \item \textbf{Se }$\pmb{\vec{v}=0}$ allora $<\vec{v}>=<0>=\{\alpha\cdot\vec{0}| 
        \alpha \in K \} = \{\vec{0}\}$\\
        e $<\vec{v}>$ ha un unico elemento.\\\\
        Se $\pmb{\vec{v}\neq \vec{0}}$ allora $<\vec{v}>= 
        \{\alpha\vec{v}|\alpha\in K\}$
        ha tanti elementi quanti sono gli elementi di $K$\\
        Per vederlo provo che 
        $ 
        \begin{cases}
            \alpha\vec{v}=\beta\vec{v}\\
            \vec{v}\neq 0\\
            \alpha,\beta\in K
        \end{cases}
        \Longleftrightarrow
        \alpha = \beta
        $\\
        ``$\Longleftarrow$" ovvio\\
        ``$\Longrightarrow$" $\alpha\vec{v}=\beta\vec{v}\Rightarrow\alpha\vec{v}
        -\beta\vec{v}=\vec{0}\Rightarrow
        \begin{cases}
            (\alpha-\beta)\vec{v}\\
            \vec{v}\neq \vec{0}
        \end{cases}
        \Longrightarrow \alpha - \beta =0 \Longrightarrow\alpha=\beta$
\end{enumerate}

% Lezione 29
\paragraph{NB} Sia $V$ uno spazio vettoriale su 
$K\in\{\mathbb{R}, \mathbb{C}\}$, 
allora: 
\begin{enumerate}
    \item $Z\leq U\leq V \Longrightarrow Z\leq V$ \footnote{
            dove $leq$ sta per \textit{sottospazio di }
        }
    \item $\{\vec{0}\}\leq V$, $V\leq V$
\end{enumerate}
Quindi se $V$ è uno spazio vettoriale su $K\in\{\mathbb{R},\mathbb{C}\} $ 
ed $U$ è un sottospazio di $V$ allora 
\begin{itemize}
    \item o $U=\{\vec{0}\}$ ed allora $|U|=1$
    \item o $U\neq\{\vec{0}\}$ ed allora $\exists\vec{u}\in U,\vec{u}\neq 0$
\end{itemize}
Essendo $U$ un sottospazio di $V$ ed $\vec{u}\in U$ allora $\alpha\vec{u}\in U$ 
$\forall\alpha\in k$\\

\[
    \begin{cases}
        <u>=\{\alpha\vec{u}| \alpha\in K\}\subseteq U\\
        \vec{u}\neq \vec{0}\Longrightarrow |<\vec{u}>|=\infty
    \end{cases}
    =|U|=\infty
\]
Sia $V$ uno spazio vettoriale du $k\in\{\mathbb{R}, \mathbb{C}\}$ \\
La \textbf{combinazione lineare degli $\pmb{n}$ vettori}
{\color{red} è una ``lista" di vettori: i vettori non sono necessariamente
distinti tra loro (possono esserci ripetizioni)}\\
$\vec{v}_1, \vec{v}_2, ..., \vec{v}_n \in V$ con \textbf{coefficienti} o pesi\\
$\alpha_1, \alpha_2, \dots, \alpha_n\in K$ è il vettore \[
    \alpha_1\vec{v}_1+\alpha_2\vec{v}_2+\alpha_3\vec{v}_3+\dots+\alpha_n\vec{v}_n\in V
\]
\paragraph{Esempio} $V=\mathbb{R}^3$, $K=\mathbb{R}$\\
\[
    \vec{v}_1= 
    \begin{bmatrix}
        1\\2\\1
    \end{bmatrix}, 
    \vec{v}_2= 
    \begin{bmatrix}
        1\\0\\0
    \end{bmatrix}, 
    \vec{v}_3= 
    \begin{bmatrix}
        1\\1\\0
    \end{bmatrix}, 
    \vec{v}_4= 
    \begin{bmatrix}
        1\\0\\0
    \end{bmatrix}
\]
\begin{align*}
    \vec{v}= 
    \begin{bmatrix}
        27\\24\\12
    \end{bmatrix}
    &
    =
    \begin{bmatrix}
        12\\24\\12
    \end{bmatrix}
    +
    \begin{bmatrix}
        12\\0\\0
    \end{bmatrix}
    +
    \begin{bmatrix}
        0\\0\\0
    \end{bmatrix}
    +
    \begin{bmatrix}
        3\\0\\0
    \end{bmatrix}\\
    &= 
    12\vec{v}_1 + 12\vec{v}_2+0\vec{v}_3+3\vec{v}_4\\
    \vec{v} & = 12\vec{v}_1 + 15\vec{v}_2 + 0\vec{v}_3+ 0\vec{v}_4
\end{align*}
$\vec{v}=\sum_{i=1}^{n}\alpha i\vec{v}i$ con\\
$
    \alpha_1=12\\
    \alpha_2=12\\
    \alpha_3=0\\
    \alpha_4=3
$\\

Dati $\vec{v}_1,\vec{v}_2, \dots, \vec{v}_n\in V$ ($V$ spazio vettoriale
su $K$) l' insieme di tutte le loro combinazioni lineari è: 
\[
    \{\alpha_1\vec{v}_1 + \alpha_2\vec{v}_2+\dots+\alpha_n\vec{v}_n\big|\alpha_1,\alpha_2,\dots
    ,\alpha_n\in K\}=
\]
    
\[
    =\Big\{\sum_{i=1}^{n}\alpha_i\vec{v}_i\big|\alpha_1,\alpha_2,\dots,\alpha_n\in K\Big\}
\]
Si indica $\pmb{<\vec{v}_1,\vec{v}_2,\dots,\vec{v}_n>}$\\
oppure $\pmb{Span(\vec{v}_1,\vec{v}_2,\dots,\vec{v}_n)}$\\
Si chiama il \textbf{sottospazio (di $\pmb{V}$) generato da }
$\pmb{\vec{v}_1, \vec{v}_2,\dots,\vec{v}_n}$

\paragraph{NB} $<\vec{v}_1,\dots,\vec{v}_n>$ è effettivamente un sottospazio di $V$\\
Infatti: 
\begin{enumerate}
    \item $\vec{0}=0\cdot\vec{v}_1+0\cdot\vec{v}_2+\dots+0\cdot\vec{v}_n$\\
        $\underset{?}{\exists}\alpha_1,\dots, \alpha_n\in K|\vec{0}=\alpha_1\vec{v}_1 
        \alpha_2\vec{v}_2+\dots+\alpha_n\vec{v}_n$\\
        {\color{red} Sì, prendiamo
        $ 
            \alpha_1=\alpha_2=\dots=\alpha_n=0
        $ 
        }
    \item 
        $ 
        \sum_{i=1}^{n}\alpha_i\vec{v}_i, 
        \sum_{i=1}^{n}\beta_i\vec{v}_i 
        \underset{?}{\Longrightarrow}
        \sum_{i=1}^{n}\alpha_i\vec{v}_i + 
        \sum_{i=1}^{n}\beta_i\vec{v}_i 
        \underset{?}{=}
        \sum_{i=1}^{n}\delta_i\vec{v}_i 
        $
    \item $\beta\in K$, $\sum_{i=1}^{n}\alpha_i\vec{v}_i
        \underset{?}{\Longrightarrow}
        \beta(\sum_{i=1}^{n}\alpha_i\vec{v}_i)\underset{?}{=}
        \sum_{i=1}^{n}\delta_i\vec{v}_i$
\end{enumerate}
\paragraph{Def} Si dice che $\vec{v}_1,\dots, \vec{v}_n$ è un 
\textbf{sistema di generatori di $\pmb{V}$}\\
$\{\vec{v}_1,\dots, \vec{v}_n\}$ \footnote{
    userò le parentesi graffe anche se i vettori $\vec{v}_1,\dots,\vec{v}_n$
    potrebbero essere tutti distinti
    } è un \textbf{insieme di generatori} 
se $V=<\vec{v}_1,\dots, \vec{v}_n>$
\\\\
$S=\{\vec{v}_1,\dots\vec{v}_n\}$ è un \textbf{inieme di generatori di $\pmb{V}$}\\
$\Longleftrightarrow <\vec{v}_1,\dots,\vec{v}_n>=\{\sum \alpha_1\vec{v}_i| \alpha_1, 
\dots, \alpha_n\in K\}\supseteq$ \footnote{
        dal momento che è sempre vero
        (qualunque sia $S$ che $\vec{v}_1,\dots\vec{v}_n\subseteq V$)
    }$ V$\\\\
{\color{red}
    $S = \{\vec{v}_1,\dots\vec{v}_n\}$ è un insieme di generatori di $V \Longleftrightarrow$\\
    $\forall\vec{v}\in V$  $\exists\alpha_1,\alpha_2, \dots, \alpha_n\in K|$
    $\vec{v}=\alpha_1\vec{v}_1+\alpha_2\vec{v}_2+\dots+\alpha_n\vec{n}$
}
\paragraph{Esempi} 
\begin{enumerate}
    \item $V=\mathbb{R}^n$, $K=\mathbb{R}$\\
        Siano $\vec{c}_1, \vec{c}_2,\dots, \vec{c}_n$  le colonne di $I_n$\\
        $S=\{\vec{c}_1, \vec{c}_2,\dots, \vec{c}_n\}$ è un insieme di generatori di $V$ \\
        $\forall \vec{v}= 
        \begin{bmatrix}
            \alpha_1\\\alpha_2\\:\\\alpha_n
        \end{bmatrix}
        \in\mathbb{R}
        $  $ 
        \exists\alpha_1, \alpha_2,\dots, \alpha_n \in\mathbb{R}|
        $
        \begin{align*}
            \begin{bmatrix}
                \alpha_1\\\alpha_2\\:\\\alpha_n
            \end{bmatrix}
            & \overset{?}{=}
            \alpha_1\vec{c}_1+
            \alpha_2\vec{c}_2+\dots+
            \alpha_n\vec{c}_n\\
            & =
            \alpha_1 
            \begin{bmatrix}
                1\\0\\0\\:\\0
            \end{bmatrix}+
            \alpha_2 
            \begin{bmatrix}
                0\\1\\0\\:\\0
            \end{bmatrix}+\dots+
            \alpha_n 
            \begin{bmatrix}
                0\\0\\0\\:\\1
            \end{bmatrix}
            =
            \begin{bmatrix}
                \alpha_1\\\alpha_2\\:\\\alpha_n
            \end{bmatrix}
        \end{align*}
        Sì: $\alpha_i=a_i$  $\forall i=1,\dots,n$
    \item $V=\mathbb{C}_n[x], K=\mathbb{C}, S=\{1,x,x^2,\dots,x^n\}$\\
        Posto $f(x)\leq n$ \\
        $\forall f(x)\in\mathbb{C}_n[x]$ $f(x)=a_0+a_1x+\dots+a_nx^n$\\
        $\underset{?}{\exists}\alpha_0, \alpha_1,\dots,\alpha_n\in\mathbb{C}$ 
        $\big|$ $ f(x)=\alpha_0\cdot 1 + \alpha_1x_1+\dots+\alpha_nx^n$\\
        Sì: $\alpha_i=a_i$
    \item $V=M_2(\mathbb{C})$ spazio vettoriale $k=\mathbb{C}$ \\\\
        $S=\Bigg\{    
            \begin{bmatrix}
                1 & 0 \\
                0 & 0
            \end{bmatrix}
            ,
            \begin{bmatrix}
                0 & 1 \\
                0 & 0
            \end{bmatrix}
            ,
            \begin{bmatrix}
                0 & 0 \\
                1 & 0
            \end{bmatrix}
            ,
            \begin{bmatrix}
                0 & 0 \\
                0 & 1
            \end{bmatrix}
        \Bigg\}$\\\\
        è un insieme di generatori di $V$\\
        $\forall 
        \begin{bmatrix}
            a & b \\
            c & d
        \end{bmatrix}
        \in M_2\mathbb{C}
        \underset{?}{\exists}\alpha_1, \alpha_2, \alpha_3, \alpha_4 \in \mathbb{C}$\\
        tali che
        \begin{align*}
            \begin{bmatrix}
                a & b \\
                c & d
            \end{bmatrix}
            & = 
            \alpha_1
            \begin{bmatrix}
                1 & 0\\
                0 & 0
            \end{bmatrix}
            \alpha_2
            \begin{bmatrix}
                0 & 1\\
                0 & 0
            \end{bmatrix}
            \alpha_3
            \begin{bmatrix}
                0 & 0\\
                1 & 0
            \end{bmatrix}
            \alpha_4
            \begin{bmatrix}
                0 & 0\\
                0 & 1
            \end{bmatrix}
            \\
            & = 
            \begin{bmatrix}
                \alpha_1 & \alpha_2\\
                \alpha_3 & \alpha_4
            \end{bmatrix}
        \end{align*}
        Sì: $\alpha_1=a$, $\alpha_2=b$, $\alpha_3=c$, $\alpha_4=d$
\end{enumerate}
{\color{red}
    Il problema di stabilire se $S$ è un insieme di generatopri di $V$ si traduce
    nel problema di stabilire se una famiglia di sistemi lineari A$\vec{x}=\vec{b}$ 
    dove A è fissato e $\vec{b}$ è un vettore dai termini noti \textbf{variabile}\\
    abbia o non abbia soluzioni. \\
}
Cioè:\\
$A\vec{x}=\vec{b}$ ha soluzioni {\color{red}$\forall\vec{b}\in\mathbb{C}^m$} ?
\footnote{ 
$A = m\times n$
}\\
$[\textrm{A}|\vec{b}] \overset{EG}{\rightarrow}[U|\vec{d}]$
{\color{red}$\vec{d}$ è libera $\forall\vec{b}\in\mathbb{C}^m$} ?
\paragraph{Esempio} $V=M_2(\mathbb{C}), K=\mathbb{C}$\\
$S=\Bigg\{    
    \begin{bmatrix}
        1 & 1 \\
        0 & 0
    \end{bmatrix}
    ,
    \begin{bmatrix}
        1 & 0 \\
        0 & 0
    \end{bmatrix}
    ,
    \begin{bmatrix}
        1 & 0 \\
        1 & 0
    \end{bmatrix}
    ,
    \begin{bmatrix}
        0 & 0 \\
        1 & 0
    \end{bmatrix}
\Bigg\}$\\\\
È uun insieme di generatori di $V$?\\
$\forall 
    \begin{bmatrix}
        a & b \\
        c & d
    \end{bmatrix}
\in M_2\mathbb{C}
\underset{?}{\exists}\alpha_1, \alpha_2, \alpha_3, \alpha_4 \in \mathbb{C}$\\\\
tali che
\begin{align*}
    \begin{bmatrix}
        a & b \\
        c & d
    \end{bmatrix}
    & = 
    \alpha_1
    \begin{bmatrix}
        1 & 1\\
        0 & 0
    \end{bmatrix}
    \alpha_2
    \begin{bmatrix}
        1 & 0\\
        0 & 0
    \end{bmatrix}
    \alpha_3
    \begin{bmatrix}
        1 & 0\\
        1 & 0
    \end{bmatrix}
    \alpha_4
    \begin{bmatrix}
        0 & 0\\
        1 & 1
    \end{bmatrix}
    \\
    & = 
    \begin{bmatrix}
        \alpha_1+\alpha_2 + \alpha_3 &  \alpha_1\\
            \alpha_3+\alpha_4 &  0
    \end{bmatrix}
\end{align*}
\color{blue}
Quindi il problema diventa:\\
è vero che il sistema lineare 
$
\begin{cases}
    \alpha_1+\alpha_2+ \alpha_3 = a\\
    \alpha_1 = b\\
    \alpha_3 + \alpha_4 = c\\
    0 = d
\end{cases}
$\\
Ha soluzioni 
\color{red}
$\forall a,b,c,d \in \mathbb{C}$ ?\\
\color{blue}
$ 
    \forall 
    \begin{bmatrix}
        a\\b\\c\\d
    \end{bmatrix}
$
Vettori di termini noti 
(quindi un vettore di termini noti variabile)
\color{black}

$$
\textrm{A}\vec{x}=\vec{b} \hspace{0.7cm} = [\textrm{A}|\vec{b}]=
\left(
\begin{array}{cccc|c}
    1 & 1 & 1 & 0 & a\\
    1 & 0 & 0 & 0 & b\\
    0 & 0 & 1 & 1 & c\\
    0 & 0 & 0 & 0 & d\\
\end{array}
\right)
\underset{EG}{\longrightarrow}
$$

\[
    [\textrm{A}|\vec{b}]= 
    \left(
    \begin{array}{cccc|c}
        1 & 1 & 1 & 0 & a\\
        1 & 0 & 0 & 0 & b\\
        0 & 0 & 1 & 1 & c\\
        0 & 0 & 0 & 0 & d\\
    \end{array}
    \right)
    \underset{E_{21}(-1)}{\longrightarrow}
    \left(
    \begin{array}{cccc|c}
        1 & 1 & 1 & 0 & a\\
        0 & -1 & -1 & 0 & b-a\\
        0 & 0 & 1 & 1 & c\\
        0 & 0 & 0 & 0 & d\\
    \end{array}
    \right)
    \underset{E_{2}(-1)}{\longrightarrow}
    \left(
    \begin{array}{cccc|c}
        1 & 1 & 1 & 0 & a\\
        0 & 1 & 1 & 0 & a-b\\
        0 & 0 & 1 & 1 & c\\
        0 & 0 & 0 & 0 & d\\
    \end{array}
    \right)
\]
La colonna $d$ non è libera $\forall a, b, c, d$ (basta prendere $d=0$)\\
$S$ non è un insieme di generatori di $V$ \\

\paragraph{NB} Abbiamo definito \textit{insieme di generatori S} \\
solo nel caso $S$ non sia una lista finita. \\
\paragraph{Def} Uno spazio vettoriale $\pmb{V}$ si dice \textbf{finitamente generato (f.g.)}
se ha un insieme di generatori che è un insieme finito. 
\footnote{Noi, in realtà abbiamo parlato solo di insiemi di generatori finiti}.\\
Esempi di spazi f.g. 

\[
    \mathbb{R}^n, \mathbb{C}^n, \mathbb{C}^n[x], \mathbb{R}^n[x], 
    \textrm{M}_{m\times n }(\mathbb{C}), \textrm{M}_{m\times n}(\mathbb{R})
\]

\paragraph{NB} non tutti gli spazi vettoriali sono finitamente generati.\\
Esempio\\
$\mathbb{C}[x], R[x]$ \textbf{non} sono finitamente generati. \footnote 
{anche gli spazi di funzioni non sono finitamente generati}.\\
Nel nostro caso, d'ora in poi, supporremmo $V$ finitamente generato. 

\paragraph{Proprietà degli insiemi di generatori:} se $V$ è uno spazio vettoriale su 
$K\in\{\mathbb{R}, \mathbb{C}\}$ 
\begin{enumerate}
    \item Sovrainsiemi di insiemi di generatori sono insiemi di generatori. \\
        Cioè: 
        $ 
        \begin{cases}
            \textrm{A insieme di generatori di } V\\
            \textrm{A} \subseteq \mathcal{B}
        \end{cases}
        \Longrightarrow \mathcal{B}
        $
        insieme di generatoridi $V$
        \textbf{Dim}
        {\color{purple}\\\\PAOLO\\\\} 
    \item Se da un insieme di generatori $S$ di $V$ si toglie un vettore che è 
        combinazione lineare dei rimanenti vettori di $S$ si ottiene un insieme 
        di vettori che è ancora un insieme di generatori di $V$
        \textbf{Esempio:}\\
        $V=< 
        \vec{v}_1=
        \begin{bmatrix}
            1\\0\\0\\0
        \end{bmatrix}
        ,
        \vec{v}_2=
        \begin{bmatrix}
            1\\1\\0\\0
        \end{bmatrix}
        ,
        \vec{v}_3=
        \begin{bmatrix}
            0\\1\\0\\0
        \end{bmatrix}
        ,
        \vec{v}_3=
        \begin{bmatrix}
            0\\0\\1\\0
        \end{bmatrix}
        >\leq \mathbb{C}^4
        $\\
        \\
        $S=\{\vec{v}_1,\vec{v}_2, \vec{v}_3, \vec{v}_4\}$ 
        è un insime di generatori di $V$\\
        notiamo anche che \\
        $\vec{v}_2=\vec{v}_1+\vec{v}_3=1\cdot\vec{v}_1+1\cdot\vec{v}_3+0\cdot\vec{v}_4 
        \Longrightarrow
        S_1=\{\vec{v}_1, \vec{v}_3, \vec{v}_4\}$\\ è ancora un insieme di generatori di $V$
        \color{blue}
        \paragraph{NB} $\vec{v}_1=\vec{v}_2-\vec{v}_3\Longrightarrow S_2=\{\vec{v}_2, \vec{v}_3, \vec{v}_4\}$
        \\è sempre un insieme di generatori di $V$
        \color{green}
        \paragraph{NB} $\vec{v}_3=-\vec{v}_1-\vec{v}_2\Longrightarrow S_3=\{\vec{v}_1, \vec{v}_2, \vec{v}_4\}$
        \\è sempre un insieme di generatori di $V$
        \color{red}
        \paragraph{Attenzione!} Invece, togliendo $\vec{v}_4$ da $S$ non si ottiene più un insieme
        di generatori di $V$.
        \color{black}
        Per quanto riguarda lo spazio vettoriale $V=\{\vec{0}\}$ si ha che 
        $S_1=\{\vec{0}\}$è un suo insieme di generatori. 
        \paragraph{NB} Per convenzione si pone che anche $S_2=\varnothing$ è un insieme di 
        generatori di $\{\vec{0}\}$
\end{enumerate}

\subsection{Insiemi di vettori linearmente indipendenti (L.I.) e 
insiemi di vettori linearmente dipendenti}
Sia $V$ uno spazio vettoriale su $K\in \{\mathbb{R}, \mathbb{C}\}$
e \\
A$\{\vec{v})_1, \vec{v_2},\dots \vec{v}_n\}$ un'``insieme" di vettori di $V$
\footnote{In realtà non è un insieme: è una lista (ci possono essere ripetizioni)}\\
\paragraph{Def}A si dice \textbf{linearmente indipendente (L.I.)} 
se l'unica combinazione lineare dei suoi elementi \textbf{nulla} è quella
con i coefficienti tutti nulli, cioè\\

$
\begin{cases}
\alpha_1\vec{v}_1+\alpha_2\vec{v}_2+...+\alpha_n\vec{v}_n\\
\alpha_1,\alpha_2,...,\alpha_n\in K
\end{cases}
$
$\Longrightarrow$
$\alpha_1=\alpha_2=...=\alpha_n=0$

\paragraph{Def 2} A$\{\vec{v}_1,\dots, \vec{v}_n\}$ si dice 
\textbf{lienarmente dipendente (L.D.)} se \textbf{non è} linearmente indipendente\\
Cioè $\exists \alpha_1,\dots, \alpha_n$ \textbf{non tutti nulli} tali che \\
$\alpha_1\vec{v}_1, \alpha_2\vec{v}_2,\dots, \alpha_n\vec{v}_n=0$\\

\paragraph{NB} per convenzione $\varnothing$ è L.I.
\paragraph{NB} $v\in V$\\
$\{\vec{v}\}$ è L.I. $\Longleftrightarrow \vec{v}=0$\\
Proviamo ``$\Longleftarrow$": ipotesi $\vec{v}_0$, tesi $\vec{v}$ L.D.\\
\paragraph{Dim} Dobbiamo provare che esiste una combinazione lineare nulla di $\vec{v}_0$\\
con coefficienti nokn tutti nulli. Eccola: prendo $\alpha=1\neq - $ coefficiente non nullo, 
ed ho: $\alpha \cdot \vec{v}=1\cdot \vec{v}= 1\cdot\vec{0} = \vec{0}$.\\
Proviamo ``$\Longrightarrow$": ipotesi $\vec{v}$ L.D., tesi $\vec{v}=\vec{0}$
\paragraph{Dim} Siccome $\{\vec{v}\}$ è L.D. $\exists\alpha\vec{v}=\vec{0}$ con 
$\alpha\neq =$. Da $\alpha\neq 0$ segue che $\exists\frac{1}{\alpha}$\\
$ 
\begin{cases}
    \alpha\vec{v}=\vec{0}\\
    \exists\frac{1}{\alpha}
\end{cases}
\Longrightarrow\frac{1}{\alpha}\cdot(\alpha\vec{v})=\frac{1}{\alpha}=\vec{0}
$\\
ma  
$ 
\frac{1}{\alpha} \cdot(\alpha\vec{v})=(\frac{1}{\alpha}\cdot \alpha)\cdot \vec{v}
=1\cdot{\vec{v}}= \vec{v}
$ e $ 
\frac{1}{\alpha}\cdot\vec{0}=0$ quindi $\vec{v}=\vec{0}$

{\color{red}
    \paragraph{NB}$\{\vec{v}\}L.D. \Longleftrightarrow\vec{v}=\vec{0}$
    \paragraph{NB}$\{\vec{v}\}L.I. \Longleftrightarrow\vec{v}\neq\vec{0}$
    \footnote{questo nb è equivalente a quello prima}
}
\subsection{Proprietà degli insiemi L.D. e degli insiemi L.I.}
\begin{enumerate}
    \item \textbf{Sovrainsiemi di L.D. sono L.D.}\\\\

        Cioè 
        $ 
        \begin{cases}
            B \subseteq A\\
            B\textrm{ L.D.}
        \end{cases}
        \Longrightarrow A \textrm{ L.D.}
        $\\\\
        
        \textbf{Dim} $B$ L.D. $\exists \alpha_1,\cdot, \alpha_n$ non tutti nulli t.c.
        $\alpha_1\vec{v}_1+\dots+\alpha_n\vec{v}_n=\vec{0}$ \\
        $\Longrightarrow \exists\alpha_1,\dots, \alpha_n,\beta_1,\dots,\beta_k$ 
        non tutti nulli tali che \\
        $\alpha_1, \vec{v}_1+\dots+\alpha_n\vec{v}_n,\beta_1\vec{w}_1,\dots,\beta_k\vec{w}_k$
        $\Longrightarrow A$ L.D.
    \item \textbf{Sottoinsiemi di L.I. sono L.I.}\\\\
        
        cioè 
        $
            \begin{cases}
                B\subseteq A\\
                A \textrm{ L.I.}
            \end{cases}
            \Longrightarrow
            B\textrm{ L.I.}
        $\\\\
        \color{red}
        \textbf{Dim}
        $
        \begin{cases}
            \alpha_1\vec{v}_1+\alpha_2\vec{v}_2+\dots+\alpha_n\vec{v}_n=\vec{0}\\
            \alpha_1,\dots,\alpha_n\in K
        \end{cases}
        \underset{?}{\Longrightarrow} \alpha_1=\alpha_2=\dots =\alpha_n=0
        $\\
        \color{black}
        $\alpha_1\vec{v}_1+\dots\alpha_n\vec{v}_n+0\vec{w}_1+0\cdot\vec{w}_2+
        \dots\vec{w}_k=\vec{0}$\\
        Siccome $A$ è L.I. \textbf{tutti} i coefficienti della combinazione lineare
        in rosso devono essere $=0$. In particolare \\
        $\alpha_1=\dots=\alpha_n=0$
\end{enumerate}

\paragraph{Def} Sia $V$ uno spazio vetoriale si $K$\\
Una \textbf{BASE} di $V$ è un insieme di generatori di $V$ che sia anche L.I. 
\paragraph{Esempi}
\begin{enumerate}
    \item $V=\mathbb{C}^n, K=\mathbb{C}$\\
        $\{\vec{e}_1,\vec{e}_2, ... , \vec{e}_n\}=$insieme delle colonne di $I_n$ 
        PAOLO
\end{enumerate}


\subsection{Basi ordinate e mappe delle coordinate}
Sia $V$ uno spazio vettoriale su $K\in\{\mathbb{R}, \mathbb{C}\}$
\paragraph{Def} Una \textbf{base ordinata} di $V$ `e una base di $V$ in cui si sia fissato l' ordine degli elementi. 
\paragraph{Esempio} $V= \mathbb{R}^2, K=\mathbb{R}$\\
$
\mathcal{B}_1= \{ 
    \begin{bmatrix}
        1\\0
    \end{bmatrix}
    ,
    \begin{bmatrix}
        0\\1
    \end{bmatrix}
\mathcal{B}_2= \{ 
    \begin{bmatrix}
        0\\1
    \end{bmatrix}
    ,
    \begin{bmatrix}
        1\\0
    \end{bmatrix}
\}$

PAOLO\\\\
SIa $\mathcal{B}=\{\vec{v}_1;\vec{v}_2;\dots\vec{v}_n\}$ una \textbf{base ordinata} di $V$ e sia $\vec{v}\in V$\\
$\mathcal{B}$ `e un insieme di generatori di $V\Longrightarrow \exists\alpha_1,\alpha_2,\dots,\alpha_n\in K\Big|\vec{v}
=\sum_{i-1}^{n}\alpha_i\vec{v}i$
$\mathcal{B}$ `e L.I. \\
PAOLO\\
$\mathcal{B}$ `e ordinata \\
\large{
    $\forall \vec{v}\in V \exists!
    \begin{bmatrix}
        \alpha_1\\
        \alpha_2\\
        :\\
        \alpha_n
    \end{bmatrix}
    \in\mathbb{C}^n $
}
tale che \\
PAOLO\\

\paragraph{Def}
Siano $V$ spazi vettoriali su $K \in\{\mathbb{R}, \mathbb{C}\}$ e $\mathcal{B}=\{\vec{v}_1, \dots, \vec{v}_n\}$ una base ordinata diu $V$\\
Sia $\vec{v}\in V$.\\
Si chiama \textbf{vettore delle coordinate} del vettore $\vec{v}\in V$ rispetto alla base ordinata B il vettore 

PAOLO 

\paragraph{Esempio} $V=\mathbb{R}^2, \vec{v}= 
\begin{bmatrix}
    2\\7
\end{bmatrix}$\\
$\mathcal{B}_1=
\begin{bmatrix}
    1\\0
\end{bmatrix}
,
\begin{bmatrix}
    0\\1
\end{bmatrix}
\mathcal{B}_2=
\begin{bmatrix}
    0\\1
\end{bmatrix}
,
\begin{bmatrix}
    1\\0
\end{bmatrix}
$\\

\[
    C_{\mathcal{B}_1}( 
    \begin{bmatrix}  
        2\\7
    \end{bmatrix}  
    )
    = 
    \begin{bmatrix}
        \alpha_1\\
        \alpha_2
    \end{bmatrix}
    \in\mathbb{R}\Big| 
    \begin{bmatrix}
        2\\7
    \end{bmatrix}
    =\alpha_1
    \begin{bmatrix}
        1\\0
    \end{bmatrix}
\]

PAOLO







    



% presta attenzione quando arrivi a INSIEMI DI VETTORI LINEARMENTE INDIPENDENTI (L.I.)


