\section{Matrici e loro operazioni} 
Una \textbf{matrice} è una tabella di numeri (o di simboli) disposti in righe e colonne, detti \textbf{coefficienti} della matrice

$$
\textrm{A}= 
\begin{bmatrix}
    2 & 3 & 0\\
    1 & 4 & 1 
\end{bmatrix}\hspace{1cm}
\textrm{A}= 
\begin{pmatrix}
    2 & 3 & 0\\
    1 & 4 & 1 
\end{pmatrix}\hspace{1cm}
\textrm{A}= 
\begin{matrix}
    2 & 3 & 0\\
    1 & 4 & 1 
\end{matrix}
$$
Altri tipi di notazioni sono sbagliati, inoltre:\\

$
\begin{bmatrix}
    2 & \\
    1 & 4
\end{bmatrix}
$ non è una matrice 
\\
\\
Il numero che si trova nella $i$-esima riga e nella $j$-esima colonna si chiama 
\textbf{coefficiente} di posto $(i,j)$
\\
\\
$A$ è $m\times n$ se ha $m$ righe e $n$ colonne

{\scriptsize ($A$ ha ``dimensioni $m\times n$") }

$$
\textrm{A}= 
\begin{tabular}{c}
    {\color{red} $\rightarrow$}\\
    {\color{blue} $\rightarrow$}
\end{tabular}
\begin{bmatrix}
    2   &     \overset{\color{blue}\downarrow}{3} &  \overset{\color{red}\downarrow}{0}\\
    1   &   {\color{blue}4}                       &  1      
\end{bmatrix}\textrm{ è }2\times 3
\hspace{1cm}
\begin{tabular}{c}
    \\
    \\
    {\color{green}$\rightarrow$}
\end{tabular}
\begin{bmatrix}
    1   &   \overset{\color{green} \downarrow}{2}\\
    i   &   7\\
    0   &   {\color{green} 3}
\end{bmatrix}\textrm{ è } 3\times 2
$$
Le posizioni sono:
$$
{\color{blue}(2,2)}\hspace{1.8cm}
{\color{red}(1,3)}\hspace{1.8cm}
{\color{green}(3,2)}
$$
Le matrici si indicano con lettere latine \textbf{maiuscole in stampatello} 
\begin{center}
    A, B, C, ...
\end{center}
I Coefficienti si indicano con le lettere latine \textbf{minuscole} in corsivo
\begin{center}
    $a_{ij}=$ il coefficiente di posti $(i,j)$ di A
\end{center}
Per scrivere in modo compatto la matrice:\\

A
$=
\begin{bmatrix}
    a_{11} & a_{12} & ............. & a_{1j} & ...................... & a_{1n}\\
    a_{21} & a_{22} & ............. & a_{2j} & ...................... & a_{2n}\\
    : & & & & & :\\
    : & & & & & :\\
    a_{i1} & a_{i2} & ............. & a_{ij} & ...................... & a_{in}\\
    : & & & & & :\\
    : & & & & & :\\
    : & & & & & :\\
    a_{m1} & a_{m2} & ............. & a_{mj} & ...................... & a_{mn}\\
\end{bmatrix}
$\\\\
\textbf{La indico:}\\
$\underset{m\times n}{\textrm{A}}=(a_{ij})$ \textbf{oppure} A$=(a_{ij})_{i=1,...,m\\j=1,...,n}${\color{purple} PAOLO}%non so come mandare a capo il pedice 
\subsection{Operazioni} 
\subsubsection{Prodotto di una matrice per uno scalare}
Dato A$=(aij), 
m\times n$ e dato uno scalare $\alpha$, si definisce 
\textbf{Prodotto dello scalare $\pmb{\alpha}$ per la matrice A} la matrice
$\underset{m\times n}{\textrm{B}}=(bij)$  dove $b_{ij}=\alpha\cdot a_{ij}$
$$\textrm{si indica B}=\alpha\cdot\textrm{A}$$
\paragraph{Esempio}
$\alpha = 1-i$\hspace{1cm} 
A$=
\begin{bmatrix}
    7       &   0   &   3i\\
    1+2i    &  -i   &  -4
\end{bmatrix}
$\\\\

$\Longrightarrow \alpha\textrm{A}=(1-j)
\begin{bmatrix}
    7       &   0   &   3i\\
    1+2i    &  -i   &  -4
\end{bmatrix}
=$



\begin{center}
    $=$
    \begin{tabular}{|c|c|c|}
        \hline
        $(1-i)7$ & $(1-i)\cdot 0$ & $(1-i)\cdot 3i$\\
        \hline
        $(1-i)(1+2i)$ & $(1-i)(-i)$ & $(1-i)(-4)$\\
        \hline
    \end{tabular}
    $=$
    \begin{tabular}{|c|c|c|}
        \hline
        $7-7i$ & 0 & $3+3i$\\
        \hline
        $3+i$ & $1-i$ & $-4+41$\\
        \hline
    \end{tabular}
    \\
    \vspace{1cm}
    \begin{tabular}{c|c}
        $(1-i)7=7-7i$ & $(1-i)(1+2i)=1-i+2i+2i^2=1-i+2i +2= 3+i$\\
        $(1-i)\cdot 3i=3i-3i^2$ & $(1-i)(-i) = -i+i^2=-i-1$\\
        $=-3i-3(-1)$ & $(1-i)(-4)=-4+4i$\\
        $=3i+3$ &\\
    \end{tabular}
\end{center}


\paragraph{NB 1} \textbf{vale la legge di cancellazione} \\
$$\alpha \cdot \textrm{A} = || \Longrightarrow \alpha=0\textrm{ oppure A}=||$$
Indico con $||$ la matrice con tutti i coefficienti $=0$
\paragraph{NB 2} 
\begin{enumerate} 
    \item $\alpha$A$=$A$\alpha \hspace{3cm}\forall\alpha$ scalare $\forall$A
    \item $1\cdot$A$=$A\hspace{3cm}$\forall$A
    \item $0\cdot$A$=||$\hspace{3cm}$\forall$A
    \item $(\alpha\cdot\beta)\cdot$A$=\alpha(\beta$A$)$\hspace{2cm}$\\
        \forall \alpha , \beta$ scalari
        $\forall$A
\end{enumerate} 
\paragraph{Notazioni} $(-1)\cdot$A$=-$A\\
A$=[a_{ij}]\hspace{1cm}(-1)\cdot$A$=[(-1)a_{ij}$
$-$A si chiama \textbf{la matrice opposta della matrice A}
\subsection{Somma di due matrici}
Date almeno due matrici A$=(a_{ij})m\times n$ e B$=(b_{ij})r\times s$ 
\textbf{aventi le stesse dimensioni}, cioè 
$
\begin{cases}
    r=m\\
    s=n
\end{cases}
$
si definisce $\underset{m\times n}{\textrm{A}+\textrm{B}}=(a_{ij}+b_{ij})$ \textbf{la somma delle due matrici}

\paragraph{Esempio} Siano 
$
\underset{{\color{red}2\times 3}}{\textrm{A}}=
\begin{bmatrix}
    1+i & 3 & 2\\
    i   & 0 & 7
\end{bmatrix},
\underset{{\color{red}2\times 2}}{\textrm{B}}=
\begin{bmatrix}
    7 & 2\\
    3i & 0
\end{bmatrix},
\underset{{\color{red}2\times 3}}{\textrm{C}}=
\begin{bmatrix}
    0 & i   & 2-i\\
    i & 7+i & i
\end{bmatrix}
$\\ 

Non posso sommare A con B, né B con C, ma posso sommare A con C:\\

\begin{center}
    A$+$C$\hspace{0.5cm}=\hspace{0.5cm}$
    \begin{tabular}{|c|c|c|}
        \hline
        1+i & 3+i & 2+2-i\\
        \hline
        i+i & 7+i & 7+i\\
        \hline
    \end{tabular}
    \hspace{0.5cm}$=$\hspace{0.5cm}
    \begin{tabular}{|c|c|c|}
        \hline
        1+i & 3+i & 4-i\\
        \hline
        2i & 7+i & 7+i\\
        \hline
    \end{tabular}
\end{center}
\paragraph{Proprietà della somma} Siano A, B, C $m\times n$, $\alpha,\beta$ scalari
\begin{enumerate}
    \item A$+($B$+$C$)=($A$+$B$)+$C
    \item A$+$B$=$B$+$A
    \item A$+||=$A
    \item A$+(-$A$)=||$
    \item $\alpha($A$+$B$)=\alpha$A$+\alpha$B
    \item $(\alpha + \beta) = \alpha$A$ +\beta$A
\end{enumerate}

\subsection{Prodotto di un vettore riga per un vettore colonna}
Sono chiamati \textbf{vettori riga} matrici con una sola riga e \textbf{vettori colonna} matrici con una sola colonna.

In notazione:
$$\underline u = 
\begin{bmatrix} 
    u_1 \\ u_2 \\ : \\ : \\ u_m
\end{bmatrix} 
$$

$$
\underline{u}^T=
\begin{bmatrix}
    u_1 & u_2 & ... & u_n
\end{bmatrix}
$$

\paragraph{Il prodotto (riga per colonna)} di $\underline{v}^T=
\begin{bmatrix}
    v_1 & v_2& ... & v_n
\end{bmatrix}$
rer 
$
\underline{u}=
\begin{bmatrix}
    u_1\\u_2\\:\\u_n
\end{bmatrix}
$ è 

$$\underline{v}^T\underline{u}=
\begin{bmatrix}
    v_1 & v_2 & ... & v_n
\end{bmatrix}
\begin{bmatrix}
    u_1\\u_2\\:\\u_n
\end{bmatrix}
=v_1u_1+v_2u_2+...+v_nu_n
$$
{\color{red}
La riga \textbf{deve necessariamente avere} tanti elementi quanti ne ha la colonna
}

\paragraph{Esempio}
$
\begin{bmatrix}
    7 & 1+i & 3
\end{bmatrix}
\begin{bmatrix}
    -1 \\1-i
\end{bmatrix}
$ non esiste\\\\
\begin{align*}
    \begin{bmatrix}
        7 & 1+i & 3
    \end{bmatrix}
    \begin{bmatrix}
        -1 \\1-i \\ 2i
    \end{bmatrix}
& = -7+(1+i)(1-i)+3\cdot 2i\\
& = -7+1^2-i^2+6i\\
& = -7+1-(-1)+6i\\
& = -7+1+1+6i\\
& = -5+6i
\end{align*}
\paragraph{NB 1}  
\begin{enumerate}
    \item $\vec{v}^T\cdot \vec{0}$
    \item $\vec{u}=
        \begin{bmatrix}
            u_1\\u_2\\: \\u_n
        \end{bmatrix},
        \vec{v}^T=
        \begin{bmatrix}
            v_1 & v_2 & ...& v_n
        \end{bmatrix}\\
        \vec{v}=
        \begin{bmatrix}
            u_1\\u_2\\: \\u_n
        \end{bmatrix},
        \vec{u}^T=
        \begin{bmatrix}
            u_1 & u_2 & ...& u_n
        \end{bmatrix}
        $
\end{enumerate}
{\color{blue}
$$ 
    \vec{v}^T\vec{u}=
    \begin{bmatrix}
        v_1 & v_2 & .. v_2
    \end{bmatrix}
    \begin{bmatrix}
        u_1 \\ u_2 \\ .. u_2
    \end{bmatrix}=
    v_1u_1+v_2u_2+...+v_nu_n=$$
$$
    =u_1v_1+u_2v_2+...+u_nv_n=
    \begin{bmatrix}
        u_1 & u_2 & .. u_2
    \end{bmatrix}
    \begin{bmatrix}
        v_1 \\ v_2 \\ .. v_2
    \end{bmatrix}
    \
$$
}
\paragraph{NB 2} \textbf{non vale la legge di cancellazione}\\
Ossia\\
$\vec{u}\neq\vec{0}$ e $\vec{v}^T\vec{u}=0\nRightarrow\vec{v}^T=\vec{0}^T$\\
ed anche $\vec{v}^T\neq\vec{0}^T$ e $\vec{v}^T\vec{u}=0\nRightarrow\vec{u}=\vec{0}$

\paragraph{Esempio} 
$
\begin{bmatrix}
    1 & 0
\end{bmatrix}
\begin{bmatrix}
    0 \\1
\end{bmatrix}
=1\cdot 0 + 0\cdot 1= 0 $


\subsection{Prodotto di due matrici (riga per colonna) } 
A$_{m\times n}$, B$_{r\times s}$ Il prodotto di A e B è possibile solo se 
$$n = r$$
$$\textrm{A}_{m\times {\color{red}n}} \textrm{B}_{{\color{red}r}\times s}$$
In tal caso il prodotto 
A$_{m\times n}\cdot$ B$_{r\times s}=$ C$_{\color{red}m\times s}$\\
dove\\
$c_{ij}=(i$-esima riga di A$)\cdot(j$-esima colonna di B$)$
{\color{purple} PAOLO}

\paragraph{Esempio} 
A$_{2\times 3}=
\begin{bmatrix}
    2 & 3 & 7\\
    6 & 0 & 5 
\end{bmatrix},
$
B$_{2\times 2}=
\begin{bmatrix}
    7 & 4\\
    2 & 3
\end{bmatrix},
$
C$_{2\times 3}=
\begin{bmatrix}
    7 & 6i & 1\\
    3 & 4 & -2
\end{bmatrix},
$\\
E$_{3\times 3}=
\begin{bmatrix}
    1 & 1 & -2\\
    4 & 3i & 3\\
    0 & -1 & 2
\end{bmatrix},
$
F$_{3\times 2}=
\begin{bmatrix}
    7i & 6 + i\\
    -2 & 5 \\
    4 & -3
\end{bmatrix}
$\\
Non esiste AB, come non esiste AC.\\
Esiste però AE$_{2\times 3}$ perche il numero delle colonne di A coincide col numero di righe di E.

Per la stessa ragione esiste anche AF$_{2\times 2}$
$$
\textrm{AF}=
\begin{bmatrix}
    22+14i & 6 + 2i\\
    20 + 42i & 21+ 6i
\end{bmatrix}
$$
Calcoliamo AE\\

$
\begin{bmatrix}
    2 & 3 & 7\\
    6 & 0 & 5 
\end{bmatrix}
\begin{bmatrix}
    1 & 1 & -2\\
    4 & 3i & 3\\
    0 & -1 & 2
\end{bmatrix}
$=
\begin{tabular}{|c|c|c|}
    \hline
    14 & -5+9i & 19 \\
    \hline
    6 & 1 & -2\\
    \hline
\end{tabular}
\begin{align*}
    c_{11}=
    \begin{bmatrix}
        2 & 3 & 7
    \end{bmatrix}
    \begin{bmatrix}
        1 \\ 4 \\ 0
    \end{bmatrix} & = 2 + 12 =14\\
    c_{12}=
    \begin{bmatrix}
        2 & 3 & 7
    \end{bmatrix}
    \begin{bmatrix}
        1 \\ 3i \\ -1
    \end{bmatrix} & = 2 + 9i -7 = -5 + 9i\\
    c_{13}=
    \begin{bmatrix}
        2 & 3 & 7
    \end{bmatrix}
    \begin{bmatrix}
        -2 \\ 3 \\ 2 
    \end{bmatrix} & = -4 +9 +14 = 19\\
    c_{21}=
    \begin{bmatrix}
        6 & 0 & 5
    \end{bmatrix}
    \begin{bmatrix}
        1 \\ 4 \\ 0
    \end{bmatrix} & =6\\
    c_{22}=
    \begin{bmatrix}
        6 & 0 & 5
    \end{bmatrix}
    \begin{bmatrix}
        1 \\ 3i \\ -1
    \end{bmatrix} & = 6-5=1 \\
    c_{23}=
    \begin{bmatrix}
        6 & 0 & 5
    \end{bmatrix}
    \begin{bmatrix}
        -2 \\ 3 \\ 2 
    \end{bmatrix}& = -12+10=-2\\
\end{align*}

\paragraph{Proprietà di cui gode il prodotto}\mbox{}\\
Supponiamo che tutte le operazioni seguenti si possano fare con A, B, C matrici 
e $\alpha$ scalare 
\begin{enumerate}
    \item $\underset{s\times r}{\textrm{A}}
        \Big(
        \underset{r\times n}{\overset{r\times m}{\textrm{B}}\cdot\overset{m\times n}{C}}
        \Big)
        =
        \Big(
        \underset{s\times m}{\overset{s\times r}{\textrm{A}}\cdot\overset{r\times m}{\textrm{B}}}
        \Big)
        \underset{m\times n}{\textrm{C}}$
        proprietà associativa

    \item $\underset{r\times m}{||}\cdot\underset{m\times n}{\textrm{A}}=\underset{r\times n}{||}$
    \item Se I$_n$indica la matrice $n\times n$ allora la matrice\\

        $$\textrm{I}_n=
        \begin{bmatrix}
            {\color{red}1} & 0 & 0 & 0 & 0 & 0 & 0 & 0 \\ 
            0 & {\color{red}1} & 0 & 0 & 0 & 0 & 0 & 0 \\ 
            0 & 0 & {\color{red}1} & 0 & 0 & 0 & 0 & 0 \\ 
            0 & 0 & 0 & {\color{red}1} & 0 & 0 & 0 & 0 \\ 
            0 & 0 & 0 & 0 & {\color{red}1} & 0 & 0 & 0 \\ 
            0 & 0 & 0 & 0 & 0 & {\color{red}1} & 0 & 0 \\ 
            0 & 0 & 0 & 0 & 0 & 0 & {\color{red}1} & 0 \\ 
            0 & 0 & 0 & 0 & 0 & 0 & 0 & {\color{red}1} 
        \end{bmatrix}
        $$

        Si chiama \textbf{matrice identica di ordine }$\pmb{n}$
        \begin{center}
            I$_2=$
            \begin{tabular}{|c|c|}
                \hline
                $1$ & $0$ \\
                \hline
                $0$ & $1$\\
                \hline
            \end{tabular} \hspace{1cm}
            I$_3=$
            \begin{tabular}{|c|c|c|}
                \hline
                $1$ & $0$ & $0$\\
                \hline
                $0$ & $1$ & $0$\\
                \hline
                $0$ & $0$ & $1$\\
                \hline
            \end{tabular} \hspace{1cm}
            Eccetera...
        \end{center}
        $$\textrm{I}_M\cdot\underset{m\times  n}{\textrm{A}} = \textrm{A} = \underset{m\times n}{\textrm{I}_n}$$
    \item A$($B$+$C$)=$AB$+$AC
    \item $($A$+$B$)$C$=$AC$+$BC
    \item $\alpha($AB$)=(\alpha\cdot$A$)$B$=${\color{red}A$(\alpha\cdot$B$)$}\\
        Questo perché $\alpha$ è uno scalare.
\end{enumerate}

\paragraph{Proprietà di cui il prodotto non gode}
\begin{enumerate}
    \item \textbf{non vale la legge di cancellazione} \\
        ossia 
        $
        \begin{cases}
            \textrm{AB}=||\\
            \textrm{A}\neq||
        \end{cases}
        \nRightarrow
        $
        B
        $=||$\\
        anche 
        $
        \begin{cases}
            \textrm{AB}=||\\
            \textrm{B}\neq||
        \end{cases}
        \nRightarrow
        $
        A
        $=||$\\\\
        \textbf{Esempio}  \\\\
        AB
        $
        \begin{bmatrix}
            2 & 3\\
            4 & 6
        \end{bmatrix}
        \begin{bmatrix}
            6 & -3\\
            -4 & 2
        \end{bmatrix}=
        \begin{bmatrix}
            2\cdot 6+3\cdot (-4) & 2\cdot (-3)+3\cdot 2\\
            4\cdot 6+6\cdot (-4) & 4\cdot (-3)+6\cdot 2
        \end{bmatrix}=
        \begin{bmatrix}
            0&0\\
            0&0 

        \end{bmatrix}
        $\\\\
        Dunque AB $=||$ che A $\neq0$ e B $\neq 0$\\
        (ed anche sia A che B sono ``quadrate")
    \item \textbf{Il prodotto (righe per colonne) NON è commutativo}\\
        Cioè AB$\neq$BA
        \begin{itemize}
            \item $\exists$AB$\nRightarrow$BA\\\\
                $
                \begin{cases}
                    \textrm{A}_{x\times n}\\
                    \textrm{B}_{n\times k}
                \end{cases}
                \Longrightarrow \exists\textrm{AB}_{m\times k}
                $
                , ma se $k\neq m$  allora $\nexists$BA
            \item 
                $
                \begin{cases}
                    \exists \textrm{AB}\\
                    \exists \textrm{BA}
                \end{cases}
                \centernot\Longrightarrow
                $
                AB e BA hanno le stesse dimensioni\\\\
                dunque \\\\
                A$_{m\times n}$ e B$_{n\times m }\Longrightarrow$
                \begin{tabular}{c}
                    $\exists$AB ed è $m\times m$\\
                    $\exists$BA ed è $n\times n$
                \end{tabular}
                $\Longrightarrow$ 
                \begin{tabular}{c}
                    se $n\neq m$ allora\\
                    AB$\neq$BA
                \end{tabular}
            \item Ma anche se A e B sono entrambe $m\times m$ per cui 
                $\exists$AB$_{m\times m}$ ed $\exists$BA$_{m\times m}$, ma non è detto che AB sia uguale a BA\\\\
                \textbf{Esempio}\\\\
                AB$=
                \begin{bmatrix}
                    2 & 3\\
                    -1 & 6
                \end{bmatrix}
                \begin{bmatrix}
                    4 & 3\\
                    2 & 6
                \end{bmatrix}
                =$
                \begin{tabular}{|c|c|}
                    \hline
                    $8+6$ & $6 + 18$ \\
                    \hline
                    $-4+ 2$ & $-3+36$\\
                    \hline
                \end{tabular}
                $=$
                \begin{tabular}{|c|c|}
                    \hline
                    $14$ & $24$\\
                    \hline
                    $8$ & $32$\\
                    \hline
                \end{tabular}\\\\\\
                BA$=
                \begin{bmatrix}
                    $4$ & $3$\\
                    $2$ & $6$
                \end{bmatrix}
                \begin{bmatrix}
                    2 & 3\\
                    -1 & 6
                \end{bmatrix}
                =$
                \begin{tabular}{|c|c|}
                    \hline
                    $8-3$ & $12+18$\\
                    \hline
                    $4-6$ & $6+36$\\
                    \hline
                \end{tabular}
                $=$
                \begin{tabular}{|c|c|}
                    \hline
                    $5$ & $30$\\
                    \hline
                    $-2$ & $42$\\
                    \hline
                \end{tabular}
        \end{itemize}
\end{enumerate}

\subsection{La trasposta}
Sia A$=(a_{ij})$ $m\times n$, la \textbf{trasposta di A} è B$=(b_{ij})$ $n\times m$ tale che \\

$b_{ij}=a_{ji}$\\\\
E si indica con B$=$A$^T$
\subsection{La coniugata}








