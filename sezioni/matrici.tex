\section{Matrici e loro operazioni} 
Una \textbf{matrice} è una tabella di numeri (o di simboli) disposti in righe e colonne, detti \textbf{coefficienti} della matrice

$$
\textrm{A}= 
\begin{bmatrix}
    2 & 3 & 0\\
    1 & 4 & 1 
\end{bmatrix}\hspace{1cm}
\textrm{A}= 
\begin{pmatrix}
    2 & 3 & 0\\
    1 & 4 & 1 
\end{pmatrix}\hspace{1cm}
\textrm{A}= 
\begin{matrix}
    2 & 3 & 0\\
    1 & 4 & 1 
\end{matrix}
$$
Altri tipi di notazioni sono sbagliati, inoltre:\\

$
\begin{bmatrix}
    2 & \\
    1 & 4
\end{bmatrix}
$ non è una matrice 
\\
\\
Il numero che si trova nella $i$-esima riga e nella $j$-esima colonna si chiama 
\textbf{coefficiente} di posto $(i,j)$
\\
\\
$A$ è $m\times n$ se ha $m$ righe e $n$ colonne

{\scriptsize ($A$ ha ``dimensioni $m\times n$") }

$$
\textrm{A}= 
\begin{tabular}{c}
    {\color{red} $\rightarrow$}\\
    {\color{blue} $\rightarrow$}
\end{tabular}
\begin{bmatrix}
    2   &     \overset{\color{blue}\downarrow}{3} &  \overset{\color{red}\downarrow}{0}\\
    1   &   {\color{blue}4}                       &  1      
\end{bmatrix}\textrm{ è }2\times 3
\hspace{1cm}
\begin{tabular}{c}
    \\
    \\
    {\color{green}$\rightarrow$}
\end{tabular}
\begin{bmatrix}
    1   &   \overset{\color{green} \downarrow}{2}\\
    i   &   7\\
    0   &   {\color{green} 3}
\end{bmatrix}\textrm{ è } 3\times 2
$$
Le posizioni sono:
$$
{\color{blue}(2,2)}\hspace{1.8cm}
{\color{red}(1,3)}\hspace{1.8cm}
{\color{green}(3,2)}
$$
Le matrici si indicano con lettere latine \textbf{maiuscole in stampatello} 
\begin{center}
    A, B, C, ...
\end{center}
I Coefficienti si indicano con le lettere latine \textbf{minuscole} in corsivo
\begin{center}
    $a_{ij}=$ il coefficiente di posti $(i,j)$ di A
\end{center}
Per scrivere in modo compatto la matrice:\\

A
$=
\begin{bmatrix}
    a_{11} & a_{12} & ............. & a_{1j} & ...................... & a_{1n}\\
    a_{21} & a_{22} & ............. & a_{2j} & ...................... & a_{2n}\\
    : & & & & & :\\
    : & & & & & :\\
    a_{i1} & a_{i2} & ............. & a_{ij} & ...................... & a_{in}\\
    : & & & & & :\\
    : & & & & & :\\
    : & & & & & :\\
    a_{m1} & a_{m2} & ............. & a_{mj} & ...................... & a_{mn}\\
\end{bmatrix}
$\\\\
{\color{purple} PAOLO}\\
\subsection{Operazioni} 
\subsubsection{Prodotto di una matrice per uno scalare}
Dato A$=(aij), 
m\times n$ e dato uno scalare $\alpha$, si definisce 
\textbf{Prodotto dello scalare $\pmb{\alpha}$ per la matrice A} la matrice
$\underset{m\times n}{\textrm{B}}=(bij)$  dove $b_{ij}=\alpha\cdot a_{ij}$
$$\textrm{si indica B}=\alpha\cdot\textrm{A}$$
\paragraph{Esempio}
$\alpha = 1-i$\hspace{1cm} 
A$=
\begin{bmatrix}
    7       &   0   &   3i\\
    1+2i    &  -i   &  -4
\end{bmatrix}
$\\\\

$\Longrightarrow \alpha\textrm{A}=(1-j)
\begin{bmatrix}
    7       &   0   &   3i\\
    1+2i    &  -i   &  -4
\end{bmatrix}
=$



\begin{center}
    $=$
    \begin{tabular}{|c|c|c|}
        \hline
        $(1-i)7$ & $(1-i)\cdot 0$ & $(1-i)\cdot 3i$\\
        \hline
        $(1-i)(1+2i)$ & $(1-i)(-i)$ & $(1-i)(-4)$\\
        \hline
    \end{tabular}
    $=$
    \begin{tabular}{|c|c|c|}
        \hline
        $7-7i$ & 0 & $3+3i$\\
        \hline
        $3+i$ & $1-i$ & $-4+41$\\
        \hline
    \end{tabular}
    \\
    \vspace{1cm}
    \begin{tabular}{c|c}
        $(1-i)7=7-7i$ & $(1-i)(1+2i)=1-i+2i+2i^2=1-i+2i +2= 3+i$\\
        $(1-i)\cdot 3i=3i-3i^2$ & $(1-i)(-i) = -i+i^2=-i-1$\\
        $=-3i-3(-1)$ & $(1-i)(-4)=-4+4i$\\
        $=3i+3$ &\\
    \end{tabular}
\end{center}


\paragraph{NB 1} \textbf{vale la legge di cancellazione} \\
$$\alpha \cdot \textrm{A} = || \Longrightarrow \alpha=0\textrm{ oppure A}=||$$
Indico con $||$ la matrice con tutti i coefficienti $=0$
\paragraph{NB 2} 
\begin{enumerate} 
    \item $\alpha$A$=$A$\alpha \hspace{3cm}\forall\alpha$ scalare $\forall$A
    \item $1\cdot$A$=$A\hspace{3cm}$\forall$A
    \item $0\cdot$A$=||$\hspace{3cm}$\forall$A
    \item $(\alpha\cdot\beta)\cdot$A$=\alpha(\beta$A$)$\hspace{2cm}$\forall \alpha , \beta$ scalari
        $\forall$A
\end{enumerate} 

